Input to the Mass Source Loading (SRC) Package is read from the file that has type ``SRC6'' in the Name File.  Any number of SRC Packages can be specified for a single groundwater transport model.  The SRC Package can be used to inject or remove solute mass from model cells at a user specified rate.  

The SRC Package has a simple constraint capability whereby the specified rate will be reduced in order to meet a concentration constraint for the model cell.  Concentration constraints are entered as auxiliary variables.  If not entered, then user-specified mass injection or removal rates will be used, even if a negative rate were to result in a negative cell concentration.  Constraints work differently depending on whether the mass rate is positive or negative.  If a positive rate is specified, then that rate will be reduced, if necessary, so that the simulated cell concentration does not exceed the constraint concentration.  If a negative rate is specified, then the rate will be reduced, if necessary, so that the simulated cell concentration does not exceed the constraint concentration.  

\vspace{5mm}
\subsubsection{Structure of Blocks}
\vspace{5mm}

\noindent \textit{FOR EACH SIMULATION}
\lstinputlisting[style=blockdefinition]{./mf6ivar/tex/gwt-src-options.dat}
\lstinputlisting[style=blockdefinition]{./mf6ivar/tex/gwt-src-dimensions.dat}
\vspace{5mm}
\noindent \textit{FOR ANY STRESS PERIOD}
\lstinputlisting[style=blockdefinition]{./mf6ivar/tex/gwt-src-period.dat}
\packageperioddescription

\vspace{5mm}
\subsubsection{Explanation of Variables}
\begin{description}
\input{./mf6ivar/tex/gwt-src-desc.tex}
\end{description}

\vspace{5mm}
\subsubsection{Example Input File}
\lstinputlisting[style=inputfile]{./mf6ivar/examples/gwt-src-example.dat}

\vspace{5mm}
\subsubsection{Available observation types}
Mass Source Loading Package observations include the simulated source loading rates (\texttt{src}). The data required for each SRC Package observation type is defined in table~\ref{table:gwt-srcobstype}. The \texttt{src} observation is equal to the simulated mass source loading rate. Negative and positive values for an observation represent a loss from and gain to the GWT model, respectively.

\begin{longtable}{p{2cm} p{2.75cm} p{2cm} p{1.25cm} p{7cm}}
\caption{Available SRC Package observation types} \tabularnewline

\hline
\hline
\textbf{Stress Package} & \textbf{Observation type} & \textbf{ID} & \textbf{ID2} & \textbf{Description} \\
\hline
\endhead

\hline
\endfoot

SRC & src & cellid or boundname & -- & Mass source loading rate between the groundwater system and a mass source loading boundary or a group of  boundaries. \\
SRC & to-mvr & cellid or boundname & -- & Mass source loading rate that is available for the MVR package for a src boundary or a group of src boundaries.
\label{table:gwt-srcobstype}
\end{longtable}

\vspace{5mm}
\subsubsection{Example Observation Input File}
\lstinputlisting[style=inputfile]{./mf6ivar/examples/gwt-src-example-obs.dat}
