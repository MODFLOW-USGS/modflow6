This section describes the data files for a \mf Groundwater Transport (GWT) Model.  A GWT Model is added to the simulation by including a GWT entry in the MODELS block of the simulation name file.

There are three types of spatial discretization approaches that can be used with the GWT Model: DIS, DISV, and DISU.  The input instructions for these three packages are not described here in this section on GWT Model input; input instructions for these three packages are described in the section on GWF Model input.

The GWT Model is designed to permit input to be gathered, as it is needed, from many different files.  Likewise, results from the model calculations can be written to a number of output files. The GWT Model Listing File is a key file to which the GWT model output is written.  As \mf runs, information about the GWT Model is written to the GWT Model Listing File, including much of the input data (as a record of the simulation) and calculated results.  Details about the files used by each package are provided in this section on the GWT Model Instructions.

The GWT Model reads a file called the Name File, which specifies most of the files that will be used in a simulation. Several files are always required whereas other files are optional depending on the simulation. The Output Control Package receives instructions from the user to control the amount and frequency of output.  Details about the Name File and the Output Control Package are described in this section.

\subsection{Multi-Species Transport Simulation}
TODO: Multi-species transport not implemented yet.  Only a single species can be represented at the moment.

\subsection{Units of Length and Time}
The GWF Model formulates the groundwater flow equation without using prescribed length and time units. Any consistent units of length and time can be used when specifying the input data for a simulation. This capability gives a certain amount of freedom to the user, but care must be exercised to avoid mixing units.  The program cannot detect the use of inconsistent units.

\subsection{Steady-State Simulations}
A steady-state transport simulation is represented by a single stress period having a single time step with the storage term set to zero. Setting the number and length of stress periods and time steps is the responsibility of the Timing Module of the \mf framework.

TODO: Need to implement mixed steady state and transient periods

\subsection{Solute Mass Budget}
A summary of all inflow (sources) and outflow (sinks) of solute mass is called a mass budget.  \mf calculates a mass budget for the overall model as a check on the acceptability of the solution, and to provide a summary of the sources and sinks of mass to the flow system.  The solute mass budget is printed to the GWT Model Listing File for selected time steps.

\subsection{Time Stepping}


\newpage
\subsection{GWT Model Name File}
The SWF Model Name File specifies the options and packages that are active for a SWF model.  The Name File contains two blocks: OPTIONS  and PACKAGES. The lines in each block can be in any order.  Files listed in the PACKAGES block must exist when the program starts. 

Comment lines are indicated when the first character in a line is one of the valid comment characters.  Commented lines can be located anywhere in the file. Any text characters can follow the comment character. Comment lines have no effect on the simulation; their purpose is to allow users to provide documentation about a particular simulation. 

\vspace{5mm}
\subsubsection{Structure of Blocks}
\lstinputlisting[style=blockdefinition]{./mf6ivar/tex/swf-nam-options.dat}
\lstinputlisting[style=blockdefinition]{./mf6ivar/tex/swf-nam-packages.dat}

\vspace{5mm}
\subsubsection{Explanation of Variables}
\begin{description}
\input{./mf6ivar/tex/swf-nam-desc.tex}
\end{description}

\begin{table}[H]
\caption{Ftype values described in this report.  The \texttt{Pname} column indicates whether or not a package name can be provided in the name file.  The capability to provide a package name also indicates that the SWF Model can have more than one package of that Ftype}
\small
\begin{center}
\begin{tabular*}{\columnwidth}{l l l}
\hline
\hline
Ftype & Input File Description & \texttt{Pname}\\
\hline
DISL6 & Line Network Discretization Input File \\
DIS6 & Structured Grid Discretization Input File \\
DISV6 & Discretization by Vertices Input File \\
DFW6 & Diffusive Wave Package \\ 
CXS6 & Cross Section Package \\ 
OC6 & Output Control Option \\
IC6 & Initial Conditions Package \\
STO6 & Storage Package \\
CHD6 & Constant Head Package & * \\ 
FLW6 & Inflow Package & * \\ 
ZDG6 & Zero-Depth Gradient Package & * \\ 
CDB6 & Critical Depth Boundary Package & * \\ 
OBS6 & Observations Option \\
\hline 
\end{tabular*}
\label{table:ftype}
\end{center}
\normalsize
\end{table}

\vspace{5mm}
\subsubsection{Example Input File}
\lstinputlisting[style=inputfile]{./mf6ivar/examples/swf-nam-example.dat}



%\newpage
%\subsection{Structured Discretization (DIS) Input File}
%Discretization information for structured grids is read from the file that is specified by ``DIS6'' as the file type.  Only one discretization input file (DISU6, DISV6 or DIS6) can be specified for a model.

\vspace{5mm}
\subsubsection{Structure of Blocks}
\lstinputlisting[style=blockdefinition]{./mf6ivar/tex/gwf-dis-options.dat}
\lstinputlisting[style=blockdefinition]{./mf6ivar/tex/gwf-dis-dimensions.dat}
\lstinputlisting[style=blockdefinition]{./mf6ivar/tex/gwf-dis-griddata.dat}

\vspace{5mm}
\subsubsection{Explanation of Variables}
\begin{description}
% DO NOT MODIFY THIS FILE DIRECTLY.  IT IS CREATED BY mf6ivar.py 

\item \textbf{Block: OPTIONS}

\begin{description}
\item \texttt{length\_units}---is the length units used for this model.  Values can be ``FEET'', ``METERS'', or ``CENTIMETERS''.  If not specified, the default is ``UNKNOWN''.

\item \texttt{NOGRB}---keyword to deactivate writing of the binary grid file.

\item \texttt{xorigin}---x-position of the lower-left corner of the model grid.  A default value of zero is assigned if not specified.  The value for XORIGIN does not affect the model simulation, but it is written to the binary grid file so that postprocessors can locate the grid in space.

\item \texttt{yorigin}---y-position of the lower-left corner of the model grid.  If not specified, then a default value equal to zero is used.  The value for YORIGIN does not affect the model simulation, but it is written to the binary grid file so that postprocessors can locate the grid in space.

\item \texttt{angrot}---counter-clockwise rotation angle (in degrees) of the lower-left corner of the model grid.  If not specified, then a default value of 0.0 is assigned.  The value for ANGROT does not affect the model simulation, but it is written to the binary grid file so that postprocessors can locate the grid in space.

\end{description}
\item \textbf{Block: DIMENSIONS}

\begin{description}
\item \texttt{nlay}---is the number of layers in the model grid.

\item \texttt{nrow}---is the number of rows in the model grid.

\item \texttt{ncol}---is the number of columns in the model grid.

\end{description}
\item \textbf{Block: GRIDDATA}

\begin{description}
\item \texttt{delr}---is the column spacing in the row direction.

\item \texttt{delc}---is the row spacing in the column direction.

\item \texttt{top}---is the top elevation for each cell in the top model layer.

\item \texttt{botm}---is the bottom elevation for each cell.

\item \texttt{idomain}---is an optional array that characterizes the existence status of a cell.  If the IDOMAIN array is not specified, then all model cells exist within the solution.  If the IDOMAIN value for a cell is 0, the cell does not exist in the simulation.  Input and output values will be read and written for the cell, but internal to the program, the cell is excluded from the solution.  If the IDOMAIN value for a cell is 1, the cell exists in the simulation.  If the IDOMAIN value for a cell is -1, the cell does not exist in the simulation.  Furthermore, the first existing cell above will be connected to the first existing cell below.  This type of cell is referred to as a ``vertical pass through'' cell.

\end{description}


\end{description}

\vspace{5mm}
\subsubsection{Example Input File}
\lstinputlisting[style=inputfile]{./mf6ivar/examples/gwf-dis-example.dat}


%\newpage
%\subsection{Discretization with Vertices (DISV) Input File}
%Discretization information for DISV grids is read from the file that is specified by ``DISV6'' as the file type.  Only one discretization input file (DISV6, DISU6 or DIS6) can be specified for a model.

The approach for numbering cell and cell vertices for the DISV Package is shown in figure~\ref{fig:gwf-fig3-2}.  The list of vertices for a cell must be in clockwise order.  Closing of the cell polygon by repeating the first vertex as the last vertex is not required in the present implementation.  Internally within the program, however, the first vertex number is added to the end of the vertex list in order to close the polygon.  Thus, users have the option for whether or not to close cell polygons.

\begin{figure}[ht]
	\centering
	\includegraphics[scale=1.0]{Figures/gwf-fig3-2}
	\caption{Schematic diagram showing the vertices and cells defined using the Discretization by Vertices Package. The list of vertices used to define each cell must be in clockwise order.  From \cite{modflow6gwf}}
	\label{fig:gwf-fig3-2}
\end{figure}


\vspace{5mm}
\subsubsection{Structure of Blocks}
\lstinputlisting[style=blockdefinition]{./mf6ivar/tex/gwf-disv-options.dat}
\lstinputlisting[style=blockdefinition]{./mf6ivar/tex/gwf-disv-dimensions.dat}
\lstinputlisting[style=blockdefinition]{./mf6ivar/tex/gwf-disv-griddata.dat}
\lstinputlisting[style=blockdefinition]{./mf6ivar/tex/gwf-disv-vertices.dat}
\lstinputlisting[style=blockdefinition]{./mf6ivar/tex/gwf-disv-cell2d.dat}

\vspace{5mm}
\subsubsection{Explanation of Variables}
\begin{description}
% DO NOT MODIFY THIS FILE DIRECTLY.  IT IS CREATED BY mf6ivar.py 

\item \textbf{Block: OPTIONS}

\begin{description}
\item \texttt{length\_units}---is the length units used for this model.  Values can be ``FEET'', ``METERS'', or ``CENTIMETERS''.  If not specified, the default is ``UNKNOWN''.

\item \texttt{NOGRB}---keyword to deactivate writing of the binary grid file.

\item \texttt{xorigin}---x-position of the origin used for model grid vertices.  This value should be provided in a real-world coordinate system.  A default value of zero is assigned if not specified.  The value for XORIGIN does not affect the model simulation, but it is written to the binary grid file so that postprocessors can locate the grid in space.

\item \texttt{yorigin}---y-position of the origin used for model grid vertices.  This value should be provided in a real-world coordinate system.  If not specified, then a default value equal to zero is used.  The value for YORIGIN does not affect the model simulation, but it is written to the binary grid file so that postprocessors can locate the grid in space.

\item \texttt{angrot}---counter-clockwise rotation angle (in degrees) of the model grid coordinate system relative to a real-world coordinate system.  If not specified, then a default value of 0.0 is assigned.  The value for ANGROT does not affect the model simulation, but it is written to the binary grid file so that postprocessors can locate the grid in space.

\item \texttt{EXPORT\_ARRAY\_ASCII}---keyword that specifies input griddata arrays should be written to layered ascii output files.

\item \texttt{EXPORT\_ARRAY\_NETCDF}---keyword that specifies input griddata arrays should be written to the model output netcdf file.

\item \texttt{NCF6}---keyword to specify that record corresponds to a netcdf configuration (NCF) file.

\item \texttt{FILEIN}---keyword to specify that an input filename is expected next.

\item \texttt{ncf6\_filename}---defines a netcdf configuration (NCF) input file.

\end{description}
\item \textbf{Block: DIMENSIONS}

\begin{description}
\item \texttt{nlay}---is the number of layers in the model grid.

\item \texttt{ncpl}---is the number of cells per layer.  This is a constant value for the grid and it applies to all layers.

\item \texttt{nvert}---is the total number of (x, y) vertex pairs used to characterize the horizontal configuration of the model grid.

\end{description}
\item \textbf{Block: GRIDDATA}

\begin{description}
\item \texttt{top}---is the top elevation for each cell in the top model layer.

\item \texttt{botm}---is the bottom elevation for each cell.

\item \texttt{idomain}---is an optional array that characterizes the existence status of a cell.  If the IDOMAIN array is not specified, then all model cells exist within the solution.  If the IDOMAIN value for a cell is 0, the cell does not exist in the simulation.  Input and output values will be read and written for the cell, but internal to the program, the cell is excluded from the solution.  If the IDOMAIN value for a cell is 1 or greater, the cell exists in the simulation.  If the IDOMAIN value for a cell is -1, the cell does not exist in the simulation.  Furthermore, the first existing cell above will be connected to the first existing cell below.  This type of cell is referred to as a ``vertical pass through'' cell.

\end{description}
\item \textbf{Block: VERTICES}

\begin{description}
\item \texttt{iv}---is the vertex number.  Records in the VERTICES block must be listed in consecutive order from 1 to NVERT.

\item \texttt{xv}---is the x-coordinate for the vertex.

\item \texttt{yv}---is the y-coordinate for the vertex.

\end{description}
\item \textbf{Block: CELL2D}

\begin{description}
\item \texttt{icell2d}---is the CELL2D number.  Records in the CELL2D block must be listed in consecutive order from the first to the last.

\item \texttt{xc}---is the x-coordinate for the cell center.

\item \texttt{yc}---is the y-coordinate for the cell center.

\item \texttt{ncvert}---is the number of vertices required to define the cell.  There may be a different number of vertices for each cell.

\item \texttt{icvert}---is an array of integer values containing vertex numbers (in the VERTICES block) used to define the cell.  Vertices must be listed in clockwise order.  Cells that are connected must share vertices.

\end{description}


\end{description}

\vspace{5mm}
\subsubsection{Example Input File}
\lstinputlisting[style=inputfile]{./mf6ivar/examples/gwf-disv-example.dat}


%\newpage
%\subsection{Unstructured Discretization (DISU) Input File}
%Discretization information for unstructured grids is read from the file that is specified by ``DISU6'' as the file type.  Only one discretization input file (DISU6, DISV6 or DIS6) can be specified for a model.

The shape and position of each cell can be defined using vertices.  This information is optional and is only read if the number of vertices (NVERT) in the DIMENSIONS block is specified and is assigned a value larger than zero.  If the vertices and two-dimensional cell information is provided in this file, then this information is also written to the binary grid file.  Providing this information may be useful for other postprocessing programs that read the binary grid file.

\vspace{5mm}
\subsubsection{Structure of Blocks}
\lstinputlisting[style=blockdefinition]{./mf6ivar/tex/gwf-disu-options.dat}
\lstinputlisting[style=blockdefinition]{./mf6ivar/tex/gwf-disu-dimensions.dat}
\lstinputlisting[style=blockdefinition]{./mf6ivar/tex/gwf-disu-griddata.dat}
\lstinputlisting[style=blockdefinition]{./mf6ivar/tex/gwf-disu-connectiondata.dat}
\lstinputlisting[style=blockdefinition]{./mf6ivar/tex/gwf-disu-vertices.dat}
\lstinputlisting[style=blockdefinition]{./mf6ivar/tex/gwf-disu-cell2d.dat}

\vspace{5mm}
\subsubsection{Explanation of Variables}
\begin{description}
% DO NOT MODIFY THIS FILE DIRECTLY.  IT IS CREATED BY mf6ivar.py 

\item \texttt{length\_units}---is the length units used for this model.  Values can be ``FEET'', ``METERS'', or ``CENTIMETERS''.  If not specified, the default is ``UNKNOWN''.

\item \texttt{NOGRB}---keyword to deactivate writing of the binary grid file.

\item \texttt{xorigin}---x-position of the origin used for model grid vertices.  This value should be provided in a real-world coordinate system.  A default value of zero is assigned if not specified.  The value for \texttt{xorigin} does not affect the model simulation, but it is written to the binary grid file so that postprocessors can locate the grid in space.

\item \texttt{yorigin}---y-position of the origin used for model grid vertices.  This value should be provided in a real-world coordinate system.  If not specified, then a default value equal to zero is used.  The value for \texttt{yorigin} does not affect the model simulation, but it is written to the binary grid file so that postprocessors can locate the grid in space.

\item \texttt{angrot}---counter-clockwise rotation angle (in degrees) of the model grid coordinate system relative to a real-world coordinate system.  If not specified, then a default value of 0.0 is assigned.  The value for \texttt{angrot} does not affect the model simulation, but it is written to the binary grid file so that postprocessors can locate the grid in space.

\item \texttt{nodes}---is the number of cells in the model grid.

\item \texttt{nja}---is the sum of the number of connections and \texttt{nodes}.  When calculating the total number of connections, the connection between cell $n$ and cell $m$ is considered to be different from the connection between cell $m$ and cell $n$.  Thus, \texttt{nja} is equal to the total number of connections, including $n$ to $m$ and $m$ to $n$, and the total number of cells.

\item \texttt{nvert}---is the total number of (x, y) vertex pairs used to define the plan-view shape of each cell in the model grid.  If \texttt{nvert} is not specified or is specified as zero, then the VERTICES and CELL2D blocks below are not read.

\item \texttt{top}---is the top elevation for each cell in the model grid.

\item \texttt{bot}---is the bottom elevation for each cell.

\item \texttt{area}---is the cell surface area (in plan view).

\item \texttt{iac}---is the number of connections (plus 1) for each cell.  The sum of \texttt{iac} must be equal to \texttt{nja}.

\item \texttt{ja}---is a list of cell number (n) followed by its connecting cell numbers (m) for each of the m cells connected to cell n. The number of values to provide for cell n is \texttt{iac(n)}.  This list is sequentially provided for the first to the last cell. The first value in the list must be cell n itself, and the remaining cells must be listed in an increasing order (sorted from lowest number to highest).  Note that the cell and its connections are only supplied for the GWF cells and their connections to the other GWF cells.  Also note that the JA list input may be chopped up to have every node number and its connectivity list on a separate line for ease in readability of the file. To further ease readability of the file, the node number of the cell whose connectivity is subsequently listed, may be expressed as a negative number the sign of which is subsequently corrected by the code.

\item \texttt{ihc}---is an index array indicating the direction between node n and all of its m connections.  If $ihc=0$ -- cell $n$ and cell $m$ are connected in the vertical direction.  Cell $n$ overlies cell $m$ if the cell number for $n$ is less than $m$; cell $m$ overlies cell $n$ if the cell number for $m$ is less than $n$.  If $ihc=1$ -- cell $n$ and cell $m$ are connected in the horizontal direction.  If $ihc=2$ -- cell $n$ and cell $m$ are connected in the horizontal direction, and the connection is vertically staggered.  A vertically staggered connection is one in which a cell is horizontally connected to more than one cell in a horizontal connection.

\item \texttt{cl12}---is the array containing connection lengths between the center of cell $n$ and the shared face with each adjacent $m$ cell.

\item \texttt{hwva}---is a symmetric array of size \texttt{nja}.  For horizontal connections, entries in \texttt{hwva} are the horizontal width perpendicular to flow.  For vertical connections, entries in \texttt{hwva} are the vertical area for flow.  Thus, values in the \texttt{hwva} array contain dimensions of both length and area.  Entries in the \texttt{hwva} array have a one-to-one correspondence with the connections specified in the \texttt{ja} array.  Likewise, there is a one-to-one correspondence between entries in the \texttt{hwva} array and entries in the \texttt{ihc} array, which specifies the connection type (horizontal or vertical).  Entries in the \texttt{hwva} array must be symmetric; the program will terminate with an error if the value for \texttt{hwva} for an $n-m$ connection does not equal the value for \texttt{hwva} for the corresponding $m-n$ connection.

\item \texttt{angldegx}---is the angle (in degrees) between the horizontal x-axis and the outward normal to the face between a cell and its connecting cells (see figure 8 in the MODFLOW-USG documentation). The angle varies between zero and 360.0.  \texttt{angldegx} is only needed if horizontal anisotropy is specified in the NPF Package.  \texttt{angldegx} does not need to be specified if horizontal anisotropy is not used.  \texttt{angldegx} is of size nja; values specified for vertical connections and for the diagonal position are not used.  Note that \texttt{angldegx} is read in degrees, which is different from MODFLOW-USG, which reads a similar variable (anglex) in radians.

\item \texttt{iv}---is the vertex number.  Records in the VERTICES block must be listed in consecutive order from 1 to \texttt{nvert}.

\item \texttt{xv}---is the x-coordinate for the vertex.

\item \texttt{yv}---is the y-coordinate for the vertex.

\item \texttt{icell2d}---is the cell2d number.  Records in the CELL2D block must be listed in consecutive order from 1 to \texttt{nodes}.

\item \texttt{xc}---is the x-coordinate for the cell center.

\item \texttt{yc}---is the y-coordinate for the cell center.

\item \texttt{ncvert}---is the number of vertices required to define the cell.  There may be a different number of vertices for each cell.

\item \texttt{icvert}---is an array of integer values containing vertex numbers (in the VERTICES block) used to define the cell.  Vertices must be listed in clockwise order.



\end{description}

\vspace{5mm}
\subsubsection{Example Input File}
\lstinputlisting[style=inputfile]{./mf6ivar/examples/gwf-disu-example.dat}



\newpage
\subsection{Initial Conditions (IC) Package}
Initial Conditions (IC) Package information is read from the file that is specified by ``IC6'' as the file type.  Only one IC Package can be specified for a SWF model. 

\vspace{5mm}
\subsubsection{Structure of Blocks}
%\lstinputlisting[style=blockdefinition]{./mf6ivar/tex/swf-ic-options.dat}
\lstinputlisting[style=blockdefinition]{./mf6ivar/tex/swf-ic-griddata.dat}

\vspace{5mm}
\subsubsection{Explanation of Variables}
\begin{description}
% DO NOT MODIFY THIS FILE DIRECTLY.  IT IS CREATED BY mf6ivar.py 

\item \textbf{Block: GRIDDATA}

\begin{description}
\item \texttt{strt}---is the initial (starting) stage---that is, stage at the beginning of the SWF Model simulation.  STRT must be specified for all SWF Model simulations. One value is read for every model reach.

\end{description}


\end{description}

\vspace{5mm}
\subsubsection{Example Input File}
\lstinputlisting[style=inputfile]{./mf6ivar/examples/swf-ic-example.dat}



\newpage
\subsection{Output Control (OC) Option}
Input to the Output Control Option of the Groundwater Transport Model is read from the file that is specified as type ``OC6'' in the Name File. If no ``OC6'' file is specified, default output control is used. The Output Control Option determines how and when concentrations are printed to the listing file and/or written to a separate binary output file.  Under the default, concentration and overall transport budget are written to the Listing File at the end of every stress period. The default printout format for concentrations is 10G11.4.  The concentrations and overall transport budget are also written to the list file if the simulation terminates prematurely due to failed convergence.

Output Control data must be specified using words.  The numeric codes supported in earlier MODFLOW versions can no longer be used.

For the PRINT and SAVE options of concentration, there is no option to specify individual layers.  Whenever the concentration array is printed or saved, all layers are printed or saved.

\vspace{5mm}
\subsubsection{Structure of Blocks}
\vspace{5mm}

\noindent \textit{FOR EACH SIMULATION}
\lstinputlisting[style=blockdefinition]{./mf6ivar/tex/gwt-oc-options.dat}
\vspace{5mm}
\noindent \textit{FOR ANY STRESS PERIOD}
\lstinputlisting[style=blockdefinition]{./mf6ivar/tex/gwt-oc-period.dat}

\vspace{5mm}
\subsubsection{Explanation of Variables}
\begin{description}
% DO NOT MODIFY THIS FILE DIRECTLY.  IT IS CREATED BY mf6ivar.py 

\item \textbf{Block: OPTIONS}

\begin{description}
\item \texttt{BUDGET}---keyword to specify that record corresponds to the budget.

\item \texttt{FILEOUT}---keyword to specify that an output filename is expected next.

\item \texttt{budgetfile}---name of the output file to write budget information.

\item \texttt{BUDGETCSV}---keyword to specify that record corresponds to the budget CSV.

\item \texttt{budgetcsvfile}---name of the comma-separated value (CSV) output file to write budget summary information.  A budget summary record will be written to this file for each time step of the simulation.

\item \texttt{CONCENTRATION}---keyword to specify that record corresponds to concentration.

\item \texttt{concentrationfile}---name of the output file to write conc information.

\item \texttt{PRINT\_FORMAT}---keyword to specify format for printing to the listing file.

\item \texttt{columns}---number of columns for writing data.

\item \texttt{width}---width for writing each number.

\item \texttt{digits}---number of digits to use for writing a number.

\item \texttt{format}---write format can be EXPONENTIAL, FIXED, GENERAL, or SCIENTIFIC.

\end{description}
\item \textbf{Block: PERIOD}

\begin{description}
\item \texttt{iper}---integer value specifying the starting stress period number for which the data specified in the PERIOD block apply.  IPER must be less than or equal to NPER in the TDIS Package and greater than zero.  The IPER value assigned to a stress period block must be greater than the IPER value assigned for the previous PERIOD block.  The information specified in the PERIOD block will continue to apply for all subsequent stress periods, unless the program encounters another PERIOD block.

\item \texttt{SAVE}---keyword to indicate that information will be saved this stress period.

\item \texttt{PRINT}---keyword to indicate that information will be printed this stress period.

\item \texttt{rtype}---type of information to save or print.  Can be BUDGET or CONCENTRATION.

\item \texttt{ocsetting}---specifies the steps for which the data will be saved.

\begin{lstlisting}[style=blockdefinition]
ALL
FIRST
LAST
FREQUENCY <frequency>
STEPS <steps(<nstp)>
\end{lstlisting}

\item \texttt{ALL}---keyword to indicate save for all time steps in period.

\item \texttt{FIRST}---keyword to indicate save for first step in period. This keyword may be used in conjunction with other keywords to print or save results for multiple time steps.

\item \texttt{LAST}---keyword to indicate save for last step in period. This keyword may be used in conjunction with other keywords to print or save results for multiple time steps.

\item \texttt{frequency}---save at the specified time step frequency. This keyword may be used in conjunction with other keywords to print or save results for multiple time steps.

\item \texttt{steps}---save for each step specified in STEPS. This keyword may be used in conjunction with other keywords to print or save results for multiple time steps.

\end{description}


\end{description}

\vspace{5mm}
\subsubsection{Example Input File}
\lstinputlisting[style=inputfile]{./mf6ivar/examples/gwt-oc-example.dat}


\newpage
\subsection{Observation (OBS) Utility for a GWT Model}
GWT & concentration & cellid & -- & Concentration at a specified cell. \\
GWT & flow-ja-face & cellid & cellid & Mass flow in dimensions of mass per time between two adjacent cells.  The mass flow rate includes the contributions from both advection and dispersion if those packages are active

\newpage
\subsection{Advection (ADV) Package}
Advection (ADV) Package information is read from the file that is specified by ``ADV6'' as the file type.  Only one ADV Package can be specified for a GWE model. 

\vspace{5mm}
\subsubsection{Structure of Blocks}
\lstinputlisting[style=blockdefinition]{./mf6ivar/tex/gwe-adv-options.dat}

\vspace{5mm}
\subsubsection{Explanation of Variables}
\begin{description}
\input{./mf6ivar/tex/gwe-adv-desc.tex}
\end{description}

\vspace{5mm}
\subsubsection{Example Input File}
\lstinputlisting[style=inputfile]{./mf6ivar/examples/gwe-adv-example.dat}



\newpage
\subsection{Dispersion (DSP) Package}
Dispersion (DSP) Package information is read from the file that is specified by ``DSP6'' as the file type.  Only one DSP Package can be specified for a GWE model.  The DSP Package is based on the mathematical formulation presented for the XT3D option of the NPF Package available to represent full three-dimensional anisotropy in groundwater flow.  XT3D can be computationally expensive and can be turned off to use a simplified and approximate form of the dispersion equations.  For most problems, however, XT3D will be required to accurately represent dispersion.

\vspace{5mm}
\subsubsection{Structure of Blocks}
\lstinputlisting[style=blockdefinition]{./mf6ivar/tex/gwe-dsp-options.dat}
\lstinputlisting[style=blockdefinition]{./mf6ivar/tex/gwe-dsp-griddata.dat}

\vspace{5mm}
\subsubsection{Explanation of Variables}
\begin{description}
\input{./mf6ivar/tex/gwe-dsp-desc.tex}
\end{description}

\vspace{5mm}
\subsubsection{Example Input File}
\lstinputlisting[style=inputfile]{./mf6ivar/examples/gwe-dsp-example.dat}



\newpage
\subsection{Source and Sink Mixing (SSM) Package}
Source and Sink Mixing (SSM) Package information is read from the file that is specified by ``SSM6'' as the file type.  Only one SSM Package can be specified for a GWT model.  The SSM Package is required if the flow model has any stress packages.

The SSM Package is used to add or remove solute mass from GWT model cells based on inflows and outflows from GWF stress packages.  If a GWF stress package provides flow into a model cell, that flow can be assigned a user-specified concentration.  If a GWT stress package removes water from a model cell, the concentration of that water is typically the concentration of the cell, but a ``MIXED'' option is also included so that the user can specify the concentration of that withdrawn water.  This may be useful for representing evapotranspiration, for example.  There are several different ways for the user to specify the concentrations.  

\begin{itemize}
\item The default condition is that sources have a concentration of zero and sinks withdraw water at the calculated concentration of the cell.  This default condition is assigned to any GWF stress package that is not included in a SOURCES block or FILEINPUT block.
\item A second option is to assign auxiliary variables in the GWF model and include a concentration for each stress boundary.  In this case, the user provides the name of the package and the name of the auxiliary variable containing concentration values for each boundary.  As described below for srctype, there are multiple options for defining this behavior.
\item A third option is to prepare an SPC6 file for any desired GWF stress package.  This SPC6 file allows users to change concentrations by stress period, or to use the time-series option to interpolate concentrations by time step.  This third option was introduced in MODFLOW version 6.3.0.  Information for this approach is entered in an optional FILEINPUT block below.
\end{itemize}

\noindent The auxiliary method and the file input method can both be used for a GWT model, but only one approach can be assigned per GWF stress package.   If a flow package specified in the SOURCES or FILEINPUT blocks is also represented using an advanced transport package (SFT, LKT, MWT, or UZT), then the advanced transport package will override SSM calculations for that package.

\vspace{5mm}
\subsubsection{Structure of Blocks}
\lstinputlisting[style=blockdefinition]{./mf6ivar/tex/gwt-ssm-options.dat}
\lstinputlisting[style=blockdefinition]{./mf6ivar/tex/gwt-ssm-sources.dat}
\lstinputlisting[style=blockdefinition]{./mf6ivar/tex/gwt-ssm-fileinput.dat}

\vspace{5mm}
\subsubsection{Explanation of Variables}
\begin{description}
% DO NOT MODIFY THIS FILE DIRECTLY.  IT IS CREATED BY mf6ivar.py 

\item \textbf{Block: OPTIONS}

\begin{description}
\item \texttt{PRINT\_FLOWS}---keyword to indicate that the list of SSM flow rates will be printed to the listing file for every stress period time step in which ``BUDGET PRINT'' is specified in Output Control.  If there is no Output Control option and ``PRINT\_FLOWS'' is specified, then flow rates are printed for the last time step of each stress period.

\item \texttt{SAVE\_FLOWS}---keyword to indicate that SSM flow terms will be written to the file specified with ``BUDGET FILEOUT'' in Output Control.

\end{description}
\item \textbf{Block: SOURCES}

\begin{description}
\item \texttt{pname}---name of the package for which an auxiliary variable contains a source concentration.

\item \texttt{srctype}---type of the source.  Must be AUX or LKT

\item \texttt{auxname}---name of the auxiliary variable in the package PNAME that contains the source concentration.

\end{description}


\end{description}

\vspace{5mm}
\subsubsection{Example Input File}
\lstinputlisting[style=inputfile]{./mf6ivar/examples/gwt-ssm-example.dat}

% when obs are ready, they should go here

\newpage
\subsection{Stress Package Concentrations (SPC) -- List-Based Input}
An SPC6 file can be prepared to provide user-specified concentrations for a GWF stress package, such a Well or General-Head Boundary Package, for example.  One SPC6 file applies to one GWF stress package.  Names for the SPC6 input files are provided in the FILEINPUT block of the SSM Package.  SPC6 entries cannot be specified in the GWT name file.  Use of the SPC6 input file is an alternative to specifying stress package concentrations as auxiliary variables in the flow model stress package.  

Stress package concentrations can be specified in the SPC6 input file using lists or arrays.  List-based input is used by default, and is described here.  Array-based concentration input can be specified for the Recharge and Evapotranspiration Packages, but only if those packages use the READASARRAYS option.  Instructions for specifying array-based stress package concentrations are described in the next section.

The boundary number in the PERIOD block corresponds to the boundary number in the GWF stress period package.  Assignment of the boundary number is straightforward for the advanced packages (SFR, LAK, MAW, and UZF) because the features in these advanced packages are defined once at the beginning of the simulation and they do not change.  For the other stress packages, however, the order of boundaries may change between stress periods.  Consider the following Well Package input file, for example:

\begin{verbatim}
# This is an example of a GWF Well Package
# in which the order of the wells changes from
# stress period 1 to 2.  This must be explicitly
# handled by the user if using the SPC6 input
# for a GWT model.
BEGIN options
END options

BEGIN dimensions
  MAXBOUND  3
  BOUNDNAMES
END dimensions

BEGIN period  1
  1 77 65   -2200  SHALLOW_WELL
  2 77 65   -24.0  INTERMEDIATE_WELL
  3 77 65   -6.20  DEEP_WELL
END period

BEGIN period  2
  1 77 65   -1100  SHALLOW_WELL
  3 77 65   -3.10  DEEP_WELL
  2 77 65   -12.0  INTERMEDIATE_WELL
END period
\end{verbatim}

\noindent In this Well input file, the order of the wells changed between periods 1 and 2.  This reordering must be explicitly taken into account by the user when creating an SSMI6 file, because the boundary number in the SSMI file corresponds to the boundary number in the Well input file.  In stress period 1, boundary number 2 is the INTERMEDIATE\_WELL, whereas in stress period 2, boundary number 2 is the DEEP\_WELL.  When using this SSMI capability to specify boundary concentrations, it is recommended that users write the corresponding GWF stress packages using the same number, cell locations, and order of boundary conditions for each stress period.   In addition, users can activate the PRINT\_FLOWS option in the SSM input file.  When the SSM Package prints the individual solute flows to the transport list file, it includes a column containing the boundary concentration.  Users can check the boundary concentrations in this output to verify that they are assigned as intended.

\vspace{5mm}
\subsubsection{Structure of Blocks}
\vspace{5mm}

\noindent \textit{FOR EACH SIMULATION}
\lstinputlisting[style=blockdefinition]{./mf6ivar/tex/utl-spc-options.dat}
\lstinputlisting[style=blockdefinition]{./mf6ivar/tex/utl-spc-dimensions.dat}
\vspace{5mm}
\noindent \textit{FOR ANY STRESS PERIOD}
\lstinputlisting[style=blockdefinition]{./mf6ivar/tex/utl-spc-period.dat}

\vspace{5mm}
\subsubsection{Explanation of Variables}
\begin{description}
% DO NOT MODIFY THIS FILE DIRECTLY.  IT IS CREATED BY mf6ivar.py 

\item \textbf{Block: OPTIONS}

\begin{description}
\item \texttt{PRINT\_INPUT}---keyword to indicate that the list of spc information will be written to the listing file immediately after it is read.

\item \texttt{TS6}---keyword to specify that record corresponds to a time-series file.

\item \texttt{FILEIN}---keyword to specify that an input filename is expected next.

\item \texttt{ts6\_filename}---defines a time-series file defining time series that can be used to assign time-varying values. See the ``Time-Variable Input'' section for instructions on using the time-series capability.

\end{description}
\item \textbf{Block: DIMENSIONS}

\begin{description}
\item \texttt{maxbound}---integer value specifying the maximum number of spc cells that will be specified for use during any stress period.

\end{description}
\item \textbf{Block: PERIOD}

\begin{description}
\item \texttt{iper}---integer value specifying the starting stress period number for which the data specified in the PERIOD block apply.  IPER must be less than or equal to NPER in the TDIS Package and greater than zero.  The IPER value assigned to a stress period block must be greater than the IPER value assigned for the previous PERIOD block.  The information specified in the PERIOD block will continue to apply for all subsequent stress periods, unless the program encounters another PERIOD block.

\item \texttt{bndno}---integer value that defines the boundary package feature number associated with the specified PERIOD data on the line. BNDNO must be greater than zero and less than or equal to MAXBOUND.

\item \texttt{spcsetting}---line of information that is parsed into a keyword and values.  Keyword values that can be used to start the SPCSETTING string include: CONCENTRATION.

\begin{lstlisting}[style=blockdefinition]
CONCENTRATION <@concentration@>
\end{lstlisting}

\item \textcolor{blue}{\texttt{concentration}---is the boundary concentration. If the Options block includes a TIMESERIESFILE entry (see the ``Time-Variable Input'' section), values can be obtained from a time series by entering the time-series name in place of a numeric value. By default, the CONCENTRATION for each boundary feature is zero.}

\end{description}


\end{description}

\subsubsection{Example Input File}
\lstinputlisting[style=inputfile]{./mf6ivar/examples/utl-spc-example.dat}

% SPC array based
\newpage
\subsection{Stress Package Concentrations (SPC) -- Array-Based Input}

TODO!!

\vspace{5mm}
\subsubsection{Structure of Blocks}
\vspace{5mm}

\noindent \textit{FOR EACH SIMULATION}
\lstinputlisting[style=blockdefinition]{./mf6ivar/tex/utl-spca-options.dat}
\vspace{5mm}
\noindent \textit{FOR ANY STRESS PERIOD}
\lstinputlisting[style=blockdefinition]{./mf6ivar/tex/utl-spca-period.dat}

\vspace{5mm}
\subsubsection{Explanation of Variables}
\begin{description}
\input{./mf6ivar/tex/utl-spca-desc.tex}
\end{description}

\subsubsection{Example Input File}
\lstinputlisting[style=inputfile]{./mf6ivar/examples/utl-spca-example.dat}



\newpage
\subsection{Mobile Storage and Transfer (MST) Package}
Mobile Storage and Transfer (MST) Package information is read from the file that is specified by ``MST6'' as the file type.  Only one MST Package can be specified for a GWT model. 

\vspace{5mm}
\subsubsection{Structure of Blocks}
\lstinputlisting[style=blockdefinition]{./mf6ivar/tex/gwt-mst-options.dat}
\lstinputlisting[style=blockdefinition]{./mf6ivar/tex/gwt-mst-griddata.dat}

\vspace{5mm}
\subsubsection{Explanation of Variables}
\begin{description}
% DO NOT MODIFY THIS FILE DIRECTLY.  IT IS CREATED BY mf6ivar.py 

\item \textbf{Block: OPTIONS}

\begin{description}
\item \texttt{SAVE\_FLOWS}---keyword to indicate that MST flow terms will be written to the file specified with ``BUDGET FILEOUT'' in Output Control.

\item \texttt{FIRST\_ORDER\_DECAY}---is a text keyword to indicate that first-order decay will occur.  Use of this keyword requires that DECAY and DECAY\_SORBED (if sorption is active) are specified in the GRIDDATA block.

\item \texttt{ZERO\_ORDER\_DECAY}---is a text keyword to indicate that zero-order decay will occur.  Use of this keyword requires that DECAY and DECAY\_SORBED (if sorption is active) are specified in the GRIDDATA block.

\item \texttt{sorption}---is a text keyword to indicate that sorption will be activated.  Valid sorption options include LINEAR, FREUNDLICH, and LANGMUIR.  Use of this keyword requires that BULK\_DENSITY and DISTCOEF are specified in the GRIDDATA block.  If sorption is specified as FREUNDLICH or LANGMUIR then SP2 is also required in the GRIDDATA block.

\end{description}
\item \textbf{Block: GRIDDATA}

\begin{description}
\item \texttt{porosity}---is the mobile domain porosity, defined as the mobile domain pore volume per mobile domain volume.  Additional information on porosity within the context of mobile and immobile domain transport simulations is included in the MODFLOW 6 Supplemental Technical Information document.

\item \texttt{decay}---is the rate coefficient for first or zero-order decay for the aqueous phase of the mobile domain.  A negative value indicates solute production.  The dimensions of decay for first-order decay is one over time.  The dimensions of decay for zero-order decay is mass per length cubed per time.  decay will have no effect on simulation results unless either first- or zero-order decay is specified in the options block.

\item \texttt{decay\_sorbed}---is the rate coefficient for first or zero-order decay for the sorbed phase of the mobile domain.  A negative value indicates solute production.  The dimensions of decay\_sorbed for first-order decay is one over time.  The dimensions of decay\_sorbed for zero-order decay is mass of solute per mass of aquifer per time.  If decay\_sorbed is not specified and both decay and sorption are active, then the program will terminate with an error.  decay\_sorbed will have no effect on simulation results unless the SORPTION keyword and either first- or zero-order decay are specified in the options block.

\item \texttt{bulk\_density}---is the bulk density of the aquifer in mass per length cubed.  bulk\_density is not required unless the SORPTION keyword is specified.  Bulk density is defined as the mobile domain solid mass per mobile domain volume.  Additional information on bulk density is included in the MODFLOW 6 Supplemental Technical Information document.

\item \texttt{distcoef}---is the distribution coefficient for the equilibrium-controlled linear sorption isotherm in dimensions of length cubed per mass.  distcoef is not required unless the SORPTION keyword is specified.

\item \texttt{sp2}---is the exponent for the Freundlich isotherm and the sorption capacity for the Langmuir isotherm.

\end{description}


\end{description}

\vspace{5mm}
\subsubsection{Example Input File}
\lstinputlisting[style=inputfile]{./mf6ivar/examples/gwt-mst-example.dat}



\newpage
\subsection{Immobile Storage and Transfer (IST) Package}
Immobile Storage and Transfer (IST) Package information is read from the file that is specified by ``IST6'' as the file type.  Any number of IST Packages can be specified for a single GWT model.  This allows the user to specify triple porosity systems, or systems with as many immobile domains as necessary. 

Subsequent to MODFLOW Version 6.4.1, substantial changes were made to the input parameter definitions and conceptualization of the IST Package.  These changes are described in Chapter 9 of the MODFLOW 6 Supplemental Technical Information document that is included with the distribution.

\vspace{5mm}
\subsubsection{Structure of Blocks}
\lstinputlisting[style=blockdefinition]{./mf6ivar/tex/gwt-ist-options.dat}
\lstinputlisting[style=blockdefinition]{./mf6ivar/tex/gwt-ist-griddata.dat}

\vspace{5mm}
\subsubsection{Explanation of Variables}
\begin{description}
% DO NOT MODIFY THIS FILE DIRECTLY.  IT IS CREATED BY mf6ivar.py 

\item \textbf{Block: OPTIONS}

\begin{description}
\item \texttt{SAVE\_FLOWS}---keyword to indicate that IST flow terms will be written to the file specified with ``BUDGET FILEOUT'' in Output Control.

\item \texttt{BUDGET}---keyword to specify that record corresponds to the budget.

\item \texttt{FILEOUT}---keyword to specify that an output filename is expected next.

\item \texttt{budgetfile}---name of the binary output file to write budget information.

\item \texttt{BUDGETCSV}---keyword to specify that record corresponds to the budget CSV.

\item \texttt{budgetcsvfile}---name of the comma-separated value (CSV) output file to write budget summary information.  A budget summary record will be written to this file for each time step of the simulation.

\item \texttt{SORPTION}---is a text keyword to indicate that sorption will be activated.  Use of this keyword requires that BULK\_DENSITY and DISTCOEF are specified in the GRIDDATA block.  The linear sorption isotherm is the only isotherm presently supported in the IST Package.

\item \texttt{FIRST\_ORDER\_DECAY}---is a text keyword to indicate that first-order decay will occur.  Use of this keyword requires that DECAY and DECAY\_SORBED (if sorption is active) are specified in the GRIDDATA block.

\item \texttt{ZERO\_ORDER\_DECAY}---is a text keyword to indicate that zero-order decay will occur.  Use of this keyword requires that DECAY and DECAY\_SORBED (if sorption is active) are specified in the GRIDDATA block.

\item \texttt{CIM}---keyword to specify that record corresponds to immobile concentration.

\item \texttt{cimfile}---name of the output file to write immobile concentrations.  This file is a binary file that has the same format and structure as a binary head and concentration file.  The value for the text variable written to the file is CIM.  Immobile domain concentrations will be written to this file at the same interval as mobile domain concentrations are saved, as specified in the GWT Model Output Control file.

\item \texttt{PRINT\_FORMAT}---keyword to specify format for printing to the listing file.

\item \texttt{columns}---number of columns for writing data.

\item \texttt{width}---width for writing each number.

\item \texttt{digits}---number of digits to use for writing a number.

\item \texttt{format}---write format can be EXPONENTIAL, FIXED, GENERAL, or SCIENTIFIC.

\end{description}
\item \textbf{Block: GRIDDATA}

\begin{description}
\item \texttt{cim}---initial concentration of the immobile domain in mass per length cubed.  If CIM is not specified, then it is assumed to be zero.

\item \texttt{porosity}---porosity of the immobile domain specified as the volume of immobile pore space per volume of immobile domain.

\item \texttt{volfrac}---fraction of the cell volume that consists of this immobile domain.

\item \texttt{zetaim}---mass transfer rate coefficient between the mobile and immobile domains, in dimensions of per time.

\item \texttt{decay}---is the rate coefficient for first or zero-order decay for the aqueous phase of the immobile domain.  A negative value indicates solute production.  The dimensions of decay for first-order decay is one over time.  The dimensions of decay for zero-order decay is mass per length cubed per time.  Decay will have no effect on simulation results unless either first- or zero-order decay is specified in the options block.

\item \texttt{decay\_sorbed}---is the rate coefficient for first or zero-order decay for the sorbed phase of the immobile domain.  A negative value indicates solute production.  The dimensions of decay\_sorbed for first-order decay is one over time.  The dimensions of decay\_sorbed for zero-order decay is mass of solute per mass of aquifer per time.  If decay\_sorbed is not specified and both decay and sorption are active, then the program will terminate with an error.  decay\_sorbed will have no effect on simulation results unless the SORPTION keyword and either first- or zero-order decay are specified in the options block.

\item \texttt{bulk\_density}---is the bulk density of this immobile domain in mass per length cubed.  bulk\_density will have no effect on simulation results unless the SORPTION keyword is specified in the options block.  Bulk density is defined as the immobile domain solid mass per volume of the immobile domain. The definition for this input variable changed after version 6.4.1 was released.  Additional information on bulk density is included in the MODFLOW 6 Supplemental Technical Information document.

\item \texttt{distcoef}---is the distribution coefficient for the equilibrium-controlled linear sorption isotherm in dimensions of length cubed per mass.  distcoef will have no effect on simulation results unless the SORPTION keyword is specified in the options block.

\end{description}


\end{description}

\vspace{5mm}
\subsubsection{Example Input File}
\lstinputlisting[style=inputfile]{./mf6ivar/examples/gwt-ist-example.dat}



\newpage
\subsection{Constant Concentration (CNC) Package}
Constant Concentration (CNC) Package information is read from the file that is specified by ``CNC6'' as the file type.  Any number of CNC Packages can be specified for a single GWT model, but the same cell cannot be designated as a constant concentration by more than one CNC entry. 

\vspace{5mm}
\subsubsection{Structure of Blocks}
\vspace{5mm}

\noindent \textit{FOR EACH SIMULATION}
\lstinputlisting[style=blockdefinition]{./mf6ivar/tex/gwt-cnc-options.dat}
\lstinputlisting[style=blockdefinition]{./mf6ivar/tex/gwt-cnc-dimensions.dat}
\vspace{5mm}
\noindent \textit{FOR ANY STRESS PERIOD}
\lstinputlisting[style=blockdefinition]{./mf6ivar/tex/gwt-cnc-period.dat}
\packageperioddescription

\vspace{5mm}
\subsubsection{Explanation of Variables}
\begin{description}
\input{./mf6ivar/tex/gwt-cnc-desc.tex}
\end{description}

\vspace{5mm}
\subsubsection{Example Input File}
\lstinputlisting[style=inputfile]{./mf6ivar/examples/gwt-cnc-example.dat}

\vspace{5mm}
\subsubsection{Available observation types}
CNC Package observations are limited to the simulated constant concentration mass flow rate (\texttt{cnc}). The data required for the CNC Package observation type is defined in table~\ref{table:gwt-cncobstype}. Negative and positive values for an observation represent a loss from and gain to the GWT model, respectively.

\begin{longtable}{p{2cm} p{2.75cm} p{2cm} p{1.25cm} p{7cm}}
\caption{Available CNC Package observation types} \tabularnewline

\hline
\hline
\textbf{Model} & \textbf{Observation type} & \textbf{ID} & \textbf{ID2} & \textbf{Description} \\
\hline
\endhead

\hline
\endfoot

CNC & cnc & cellid or boundname & -- & Flow between the groundwater system and a constant-concentration boundary or a group of cells with constant-concentration boundaries.

\label{table:gwt-cncobstype}
\end{longtable}

\vspace{5mm}
\subsubsection{Example Observation Input File}
\lstinputlisting[style=inputfile]{./mf6ivar/examples/gwt-cnc-example-obs.dat}


\newpage
\subsection{Mass Source Loading (SRC) Package}
Input to the Mass Source Loading (SRC) Package is read from the file that has type ``SRC6'' in the Name File.  Any number of SRC Packages can be specified for a single groundwater transport model.

\vspace{5mm}
\subsubsection{Structure of Blocks}
\vspace{5mm}

\noindent \textit{FOR EACH SIMULATION}
\lstinputlisting[style=blockdefinition]{./mf6ivar/tex/gwt-src-options.dat}
\lstinputlisting[style=blockdefinition]{./mf6ivar/tex/gwt-src-dimensions.dat}
\vspace{5mm}
\noindent \textit{FOR ANY STRESS PERIOD}
\lstinputlisting[style=blockdefinition]{./mf6ivar/tex/gwt-src-period.dat}
\packageperioddescription

\vspace{5mm}
\subsubsection{Explanation of Variables}
\begin{description}
\input{./mf6ivar/tex/gwt-src-desc.tex}
\end{description}

\vspace{5mm}
\subsubsection{Example Input File}
\lstinputlisting[style=inputfile]{./mf6ivar/examples/gwt-src-example.dat}

\vspace{5mm}
\subsubsection{Available observation types}
Mass Source Loading Package observations include the simulated source loading rates (\texttt{src}). The data required for each SRC Package observation type is defined in table~\ref{table:gwt-srcobstype}. The \texttt{src} observation is equal to the simulated mass source loading rate. Negative and positive values for an observation represent a loss from and gain to the GWT model, respectively.

\begin{longtable}{p{2cm} p{2.75cm} p{2cm} p{1.25cm} p{7cm}}
\caption{Available SRC Package observation types} \tabularnewline

\hline
\hline
\textbf{Stress Package} & \textbf{Observation type} & \textbf{ID} & \textbf{ID2} & \textbf{Description} \\
\hline
\endhead

\hline
\endfoot

SRC & src & cellid or boundname & -- & Mass source loading rate between the groundwater system and a mass source loading boundary or a group of  boundaries. \\
SRC & to-mvr & cellid or boundname & -- & Mass source loading rate that is available for the MVR package for a src boundary or a group of src boundaries.
\label{table:gwt-srcobstype}
\end{longtable}

\vspace{5mm}
\subsubsection{Example Observation Input File}
\lstinputlisting[style=inputfile]{./mf6ivar/examples/gwt-src-example-obs.dat}


\newpage
\subsection{Streamflow Transport (SFT) Package}
Streamflow Transport (SFT) Package information is read from the file that is specified by ``SFT6'' as the file type.  There can be as many SFT Packages as necessary for a GWT model. Each SFT Package is designed to work with flows from a corresponding GWF SFR Package. \mf uses the package name to determine which SFR Package corresponds to the SFT Package.  Therefore, the package name of the SFT Package (as specified in the GWT name file) must match with the name of the corresponding SFR Package (as specified in the GWF name file).  The GWT SFT Package cannot be used without a corresponding GWF SFR Package.

The SFT Package does not have a dimensions block; instead, dimensions for the SFT Package are set using the dimensions from the corresponding SFR Package.  For example, the SFR Package requires specification of the number of reaches (NREACHES).  SFT sets the number of reaches equal to NREACHES.  Therefore, the PACKAGEDATA block below must have NREACHES entries in it.

\vspace{5mm}
\subsubsection{Structure of Blocks}
\lstinputlisting[style=blockdefinition]{./mf6ivar/tex/gwt-sft-options.dat}
\lstinputlisting[style=blockdefinition]{./mf6ivar/tex/gwt-sft-packagedata.dat}
\lstinputlisting[style=blockdefinition]{./mf6ivar/tex/gwt-sft-period.dat}

\vspace{5mm}
\subsubsection{Explanation of Variables}
\begin{description}
\input{./mf6ivar/tex/gwt-sft-desc.tex}
\end{description}

\vspace{5mm}
\subsubsection{Example Input File}
\lstinputlisting[style=inputfile]{./mf6ivar/examples/gwt-sft-example.dat}



\newpage
\subsection{Lake Transport (LKT) Package}
Lake Transport (LKT) Package information is read from the file that is specified by ``LKT6'' as the file type.  There can be as many LKT Packages as necessary for a GWT model. Each LKT Package is designed to work with flows from a single corresponding GWF LAK Package. By default \mf uses the LKT package name to determine which LAK Package corresponds to the LKT Package.  Therefore, the package name of the LKT Package (as specified in the GWT name file) must match with the name of the corresponding LAK Package (as specified in the GWF name file).  Alternatively, the name of the flow package can be specified using the FLOW\_PACKAGE\_NAME keyword in the options block.  The GWT LKT Package cannot be used without a corresponding GWF LAK Package.

The LKT Package does not have a dimensions block; instead, dimensions for the LKT Package are set using the dimensions from the corresponding LAK Package.  For example, the LAK Package requires specification of the number of lakes (NLAKES).  LKT sets the number of lakes equal to NLAKES.  Therefore, the PACKAGEDATA block below must have NLAKES entries in it.

\vspace{5mm}
\subsubsection{Structure of Blocks}
\lstinputlisting[style=blockdefinition]{./mf6ivar/tex/gwt-lkt-options.dat}
\lstinputlisting[style=blockdefinition]{./mf6ivar/tex/gwt-lkt-packagedata.dat}
\lstinputlisting[style=blockdefinition]{./mf6ivar/tex/gwt-lkt-period.dat}

\vspace{5mm}
\subsubsection{Explanation of Variables}
\begin{description}
% DO NOT MODIFY THIS FILE DIRECTLY.  IT IS CREATED BY mf6ivar.py 

\item \textbf{Block: OPTIONS}

\begin{description}
\item \texttt{flow\_package\_name}---keyword to specify the name of the corresponding flow package.  If not specified, then the corresponding flow package must have the same name as this advanced transport package (the name associated with this package in the GWT name file).

\item \texttt{auxiliary}---defines an array of one or more auxiliary variable names.  There is no limit on the number of auxiliary variables that can be provided on this line; however, lists of information provided in subsequent blocks must have a column of data for each auxiliary variable name defined here.   The number of auxiliary variables detected on this line determines the value for naux.  Comments cannot be provided anywhere on this line as they will be interpreted as auxiliary variable names.  Auxiliary variables may not be used by the package, but they will be available for use by other parts of the program.  The program will terminate with an error if auxiliary variables are specified on more than one line in the options block.

\item \texttt{flow\_package\_auxiliary\_name}---keyword to specify the name of an auxiliary variable in the corresponding flow package.  If specified, then the simulated concentrations from this advanced transport package will be copied into the auxiliary variable specified with this name.  Note that the flow package must have an auxiliary variable with this name or the program will terminate with an error.  If the flows for this advanced transport package are read from a file, then this option will have no effect.

\item \texttt{BOUNDNAMES}---keyword to indicate that boundary names may be provided with the list of lake cells.

\item \texttt{PRINT\_INPUT}---keyword to indicate that the list of lake information will be written to the listing file immediately after it is read.

\item \texttt{PRINT\_CONCENTRATION}---keyword to indicate that the list of lake concentration will be printed to the listing file for every stress period in which ``CONCENTRATION PRINT'' is specified in Output Control.  If there is no Output Control option and PRINT\_CONCENTRATION is specified, then concentration are printed for the last time step of each stress period.

\item \texttt{PRINT\_FLOWS}---keyword to indicate that the list of lake flow rates will be printed to the listing file for every stress period time step in which ``BUDGET PRINT'' is specified in Output Control.  If there is no Output Control option and ``PRINT\_FLOWS'' is specified, then flow rates are printed for the last time step of each stress period.

\item \texttt{SAVE\_FLOWS}---keyword to indicate that lake flow terms will be written to the file specified with ``BUDGET FILEOUT'' in Output Control.

\item \texttt{CONCENTRATION}---keyword to specify that record corresponds to concentration.

\item \texttt{concfile}---name of the binary output file to write concentration information.

\item \texttt{BUDGET}---keyword to specify that record corresponds to the budget.

\item \texttt{FILEOUT}---keyword to specify that an output filename is expected next.

\item \texttt{budgetfile}---name of the binary output file to write budget information.

\item \texttt{BUDGETCSV}---keyword to specify that record corresponds to the budget CSV.

\item \texttt{budgetcsvfile}---name of the comma-separated value (CSV) output file to write budget summary information.  A budget summary record will be written to this file for each time step of the simulation.

\item \texttt{TS6}---keyword to specify that record corresponds to a time-series file.

\item \texttt{FILEIN}---keyword to specify that an input filename is expected next.

\item \texttt{ts6\_filename}---defines a time-series file defining time series that can be used to assign time-varying values. See the ``Time-Variable Input'' section for instructions on using the time-series capability.

\item \texttt{OBS6}---keyword to specify that record corresponds to an observations file.

\item \texttt{obs6\_filename}---name of input file to define observations for the LKT package. See the ``Observation utility'' section for instructions for preparing observation input files. Tables \ref{table:gwf-obstypetable} and \ref{table:gwt-obstypetable} lists observation type(s) supported by the LKT package.

\end{description}
\item \textbf{Block: PACKAGEDATA}

\begin{description}
\item \texttt{ifno}---integer value that defines the feature (lake) number associated with the specified PACKAGEDATA data on the line. IFNO must be greater than zero and less than or equal to NLAKES. Lake information must be specified for every lake or the program will terminate with an error.  The program will also terminate with an error if information for a lake is specified more than once.

\item \texttt{strt}---real value that defines the starting concentration for the lake.

\item \textcolor{blue}{\texttt{aux}---represents the values of the auxiliary variables for each lake. The values of auxiliary variables must be present for each lake. The values must be specified in the order of the auxiliary variables specified in the OPTIONS block.  If the package supports time series and the Options block includes a TIMESERIESFILE entry (see the ``Time-Variable Input'' section), values can be obtained from a time series by entering the time-series name in place of a numeric value.}

\item \texttt{boundname}---name of the lake cell.  BOUNDNAME is an ASCII character variable that can contain as many as 40 characters.  If BOUNDNAME contains spaces in it, then the entire name must be enclosed within single quotes.

\end{description}
\item \textbf{Block: PERIOD}

\begin{description}
\item \texttt{iper}---integer value specifying the starting stress period number for which the data specified in the PERIOD block apply.  IPER must be less than or equal to NPER in the TDIS Package and greater than zero.  The IPER value assigned to a stress period block must be greater than the IPER value assigned for the previous PERIOD block.  The information specified in the PERIOD block will continue to apply for all subsequent stress periods, unless the program encounters another PERIOD block.

\item \texttt{ifno}---integer value that defines the feature (lake) number associated with the specified PERIOD data on the line. IFNO must be greater than zero and less than or equal to NLAKES.

\item \texttt{laksetting}---line of information that is parsed into a keyword and values.  Keyword values that can be used to start the LAKSETTING string include: STATUS, CONCENTRATION, RAINFALL, EVAPORATION, RUNOFF, and AUXILIARY.  These settings are used to assign the concentration of associated with the corresponding flow terms.  Concentrations cannot be specified for all flow terms.  For example, the Lake Package supports a ``WITHDRAWAL'' flow term.  If this withdrawal term is active, then water will be withdrawn from the lake at the calculated concentration of the lake.

\begin{lstlisting}[style=blockdefinition]
STATUS <status>
CONCENTRATION <@concentration@>
RAINFALL <@rainfall@>
EVAPORATION <@evaporation@>
RUNOFF <@runoff@>
EXT-INFLOW <@ext-inflow@>
AUXILIARY <auxname> <@auxval@> 
\end{lstlisting}

\item \texttt{status}---keyword option to define lake status.  STATUS can be ACTIVE, INACTIVE, or CONSTANT. By default, STATUS is ACTIVE, which means that concentration will be calculated for the lake.  If a lake is inactive, then there will be no solute mass fluxes into or out of the lake and the inactive value will be written for the lake concentration.  If a lake is constant, then the concentration for the lake will be fixed at the user specified value.

\item \textcolor{blue}{\texttt{concentration}---real or character value that defines the concentration for the lake. The specified CONCENTRATION is only applied if the lake is a constant concentration lake. If the Options block includes a TIMESERIESFILE entry (see the ``Time-Variable Input'' section), values can be obtained from a time series by entering the time-series name in place of a numeric value.}

\item \textcolor{blue}{\texttt{rainfall}---real or character value that defines the rainfall solute concentration $(ML^{-3})$ for the lake. If the Options block includes a TIMESERIESFILE entry (see the ``Time-Variable Input'' section), values can be obtained from a time series by entering the time-series name in place of a numeric value.}

\item \textcolor{blue}{\texttt{evaporation}---real or character value that defines the concentration of evaporated water $(ML^{-3})$ for the lake. If this concentration value is larger than the simulated concentration in the lake, then the evaporated water will be removed at the same concentration as the lake.  If the Options block includes a TIMESERIESFILE entry (see the ``Time-Variable Input'' section), values can be obtained from a time series by entering the time-series name in place of a numeric value.}

\item \textcolor{blue}{\texttt{runoff}---real or character value that defines the concentration of runoff $(ML^{-3})$ for the lake. Value must be greater than or equal to zero. If the Options block includes a TIMESERIESFILE entry (see the ``Time-Variable Input'' section), values can be obtained from a time series by entering the time-series name in place of a numeric value.}

\item \textcolor{blue}{\texttt{ext-inflow}---real or character value that defines the concentration of external inflow $(ML^{-3})$ for the lake. Value must be greater than or equal to zero. If the Options block includes a TIMESERIESFILE entry (see the ``Time-Variable Input'' section), values can be obtained from a time series by entering the time-series name in place of a numeric value.}

\item \texttt{AUXILIARY}---keyword for specifying auxiliary variable.

\item \texttt{auxname}---name for the auxiliary variable to be assigned AUXVAL.  AUXNAME must match one of the auxiliary variable names defined in the OPTIONS block. If AUXNAME does not match one of the auxiliary variable names defined in the OPTIONS block the data are ignored.

\item \textcolor{blue}{\texttt{auxval}---value for the auxiliary variable. If the Options block includes a TIMESERIESFILE entry (see the ``Time-Variable Input'' section), values can be obtained from a time series by entering the time-series name in place of a numeric value.}

\end{description}


\end{description}

\vspace{5mm}
\subsubsection{Example Input File}
\lstinputlisting[style=inputfile]{./mf6ivar/examples/gwt-lkt-example.dat}

\vspace{5mm}
\subsubsection{Available observation types}
Lake Transport Package observations include lake concentration and all of the terms that contribute to the continuity equation for each lake. Additional LKT Package observations include mass flow rates for individual outlets, lakes, or groups of lakes (\texttt{outlet}). The data required for each LKT Package observation type is defined in table~\ref{table:gwt-lktobstype}. Negative and positive values for \texttt{lkt} observations represent a loss from and gain to the GWT model, respectively. For all other flow terms, negative and positive values represent a loss from and gain from the LKT package, respectively.

\begin{longtable}{p{2cm} p{2.75cm} p{2cm} p{1.25cm} p{7cm}}
\caption{Available LKT Package observation types} \tabularnewline

\hline
\hline
\textbf{Stress Package} & \textbf{Observation type} & \textbf{ID} & \textbf{ID2} & \textbf{Description} \\
\hline
\endfirsthead

\captionsetup{textformat=simple}
\caption*{\textbf{Table \arabic{table}.}{\quad}Available LKT Package observation types.---Continued} \tabularnewline

\hline
\hline
\textbf{Stress Package} & \textbf{Observation type} & \textbf{ID} & \textbf{ID2} & \textbf{Description} \\
\hline
\endhead


\hline
\endfoot

% general APT observations
LKT & concentration & ifno or boundname & -- & Lake concentration. If boundname is specified, boundname must be unique for each lake. \\
LKT & flow-ja-face & ifno or boundname & ifno or -- & Mass flow between two lakes connected by an outlet.  If more than one outlet is used to connect the same two lakes, then the mass flow for only the first outlet can be observed.  If a boundname is specified for ID1, then the result is the total mass flow for all outlets for a lake. If a boundname is specified for ID1 then ID2 is not used.\\
LKT & storage & ifno or boundname & -- & Simulated mass storage flow rate for a lake or group of lakes. \\
LKT & constant & ifno or boundname & -- & Simulated mass constant-flow rate for a lake or group of lakes. \\
LKT & from-mvr & ifno or boundname & -- & Simulated mass inflow into a lake or group of lakes from the MVT package. Mass inflow is calculated as the product of provider concentration and the mover flow rate. \\
LKT & to-mvr & outletno or boundname & -- & Mass outflow from a lake outlet, a lake, or a group of lakes that is available for the MVR package. If boundname is not specified for ID, then the outflow available for the MVR package from a specific lake outlet is observed. In this case, ID is the outlet number, which must be between 1 and NOUTLETS. \\
LKT & lkt & ifno or boundname & \texttt{iconn} or -- & Mass flow rate for a lake or group of lakes and its aquifer connection(s). If boundname is not specified for ID, then the simulated lake-aquifer flow rate at a specific lake connection is observed. In this case, ID2 must be specified and is the connection number \texttt{iconn} for lake \texttt{ifno}. \\

%observations specific to the lake package
% rainfall evaporation runoff ext-inflow withdrawal outflow
LKT & rainfall & ifno or boundname & -- & Rainfall rate applied to a lake or group of lakes multiplied by the rainfall concentration. \\
LKT & evaporation & ifno or boundname & -- & Simulated evaporation rate from a lake or group of lakes multiplied by the evaporation concentration. \\
LKT & runoff & ifno or boundname & -- & Runoff rate applied to a lake or group of lakes multiplied by the runoff concentration. \\
LKT & ext-inflow & ifno or boundname & -- & Mass inflow into a lake or group of lakes calculated as the external inflow rate multiplied by the inflow concentration. \\
LKT & withdrawal & ifno or boundname & -- & Specified withdrawal rate from a lake or group of lakes multiplied by the simulated lake concentration. \\
LKT & ext-outflow & ifno or boundname & -- & External outflow from a lake or a group of lakes, through their outlets, to an external boundary.  If the water mover is active, the reported ext-outflow value plus the rate to mover is equal to the total outlet outflow.

%LKT & outlet-inflow & ifno or boundname & -- & Simulated inflow from upstream lake outlets into a lake or group of lakes. \\
%LKT & inflow & ifno or boundname & -- & Sum of specified inflow and simulated inflow from upstream lake outlets into a lake or group of lakes. \\
%LKT & outlet & outletno or boundname & -- & Simulate outlet flow rate from a lake outlet, a lake, or a group of lakes. If boundname is not specified for ID, then the flow from a specific lake outlet is observed. In this case, ID is the outlet number outletno. \\
%LKT & volume & ifno or boundname & -- & Simulated lake volume or group of lakes. \\
%LKT & surface-area & ifno or boundname & -- & Simulated surface area for a lake or group of lakes. \\
%LKT & wetted-area & ifno or boundname & \texttt{iconn} or -- & Simulated wetted-area for a lake or group of lakes and its aquifer connection(s). If boundname is not specified for ID, then the wetted area of a specific lake connection is observed. In this case, ID2 must be specified and is the connection number \texttt{iconn}. \\
%LKT & conductance & ifno or boundname & \texttt{iconn} or -- & Calculated conductance for a lake or group of lakes and its aquifer connection(s). If boundname is not specified for ID, then the calculated conductance of a specific lake connection is observed. In this case, ID2 must be specified and is the connection number \texttt{iconn}.

\label{table:gwt-lktobstype}
\end{longtable}

\vspace{5mm}
\subsubsection{Example Observation Input File}
\lstinputlisting[style=inputfile]{./mf6ivar/examples/gwt-lkt-example-obs.dat}




\newpage
\subsection{Multi-Aquifer Well Transport (MWT) Package}
Multi-Aquifer Well Transport (MWT) Package information is read from the file that is specified by ``MWT6'' as the file type.  There can be as many MWT Packages as necessary for a GWT model. Each MWT Package is designed to work with flows from a corresponding GWF MAW Package. By default \mf uses the MWT package name to determine which MAW Package corresponds to the MWT Package.  Therefore, the package name of the MWT Package (as specified in the GWT name file) must match with the name of the corresponding MAW Package (as specified in the GWF name file).  Alternatively, the name of the flow package can be specified using the FLOW\_PACKAGE\_NAME keyword in the options block.  The GWT MWT Package cannot be used without a corresponding GWF MAW Package.

The MWT Package does not have a dimensions block; instead, dimensions for the MWT Package are set using the dimensions from the corresponding MAW Package.  For example, the MAW Package requires specification of the number of wells (NMAWWELLS).  MWT sets the number of wells equal to NMAWWELLS.  Therefore, the PACKAGEDATA block below must have NMAWWELLS entries in it.

\vspace{5mm}
\subsubsection{Structure of Blocks}
\lstinputlisting[style=blockdefinition]{./mf6ivar/tex/gwt-mwt-options.dat}
\lstinputlisting[style=blockdefinition]{./mf6ivar/tex/gwt-mwt-packagedata.dat}
\lstinputlisting[style=blockdefinition]{./mf6ivar/tex/gwt-mwt-period.dat}

\vspace{5mm}
\subsubsection{Explanation of Variables}
\begin{description}
\input{./mf6ivar/tex/gwt-mwt-desc.tex}
\end{description}

\vspace{5mm}
\subsubsection{Example Input File}
\lstinputlisting[style=inputfile]{./mf6ivar/examples/gwt-mwt-example.dat}

\vspace{5mm}
\subsubsection{Available observation types}
Multi-Aquifer Well Transport Package observations include well concentration and all of the terms that contribute to the continuity equation for each well. Additional MWT Package observations include mass flow rates for individual wells, or groups of wells; the well volume (\texttt{volume}); and the conductance for a well-aquifer connection conductance (\texttt{conductance}). The data required for each MWT Package observation type is defined in table~\ref{table:gwt-mwtobstype}. Negative and positive values for \texttt{mwt} observations represent a loss from and gain to the GWT model, respectively. For all other flow terms, negative and positive values represent a loss from and gain from the MWT package, respectively.

\begin{longtable}{p{2cm} p{2.75cm} p{2cm} p{1.25cm} p{7cm}}
\caption{Available MWT Package observation types} \tabularnewline

\hline
\hline
\textbf{Stress Package} & \textbf{Observation type} & \textbf{ID} & \textbf{ID2} & \textbf{Description} \\
\hline
\endfirsthead

\captionsetup{textformat=simple}
\caption*{\textbf{Table \arabic{table}.}{\quad}Available MWT Package observation types.---Continued} \tabularnewline

\hline
\hline
\textbf{Stress Package} & \textbf{Observation type} & \textbf{ID} & \textbf{ID2} & \textbf{Description} \\
\hline
\endhead


\hline
\endfoot

% general APT observations
MWT & concentration & ifno or boundname & -- & Well concentration. If boundname is specified, boundname must be unique for each well. \\
%flowjaface not included
MWT & storage & ifno or boundname & -- & Simulated mass storage flow rate for a well or group of wells. \\
MWT & constant & ifno or boundname & -- & Simulated mass constant-flow rate for a well or group of wells. \\
MWT & from-mvr & ifno or boundname & -- & Simulated mass inflow into a well or group of wells from the MVT package. Mass inflow is calculated as the product of provider concentration and the mover flow rate. \\
MWT & mwt & ifno or boundname & \texttt{iconn} or -- & Mass flow rate for a well or group of wells and its aquifer connection(s). If boundname is not specified for ID, then the simulated well-aquifer flow rate at a specific well connection is observed. In this case, ID2 must be specified and is the connection number \texttt{iconn} for well \texttt{ifno}. \\

% observations specific to the mwt package
MWT & rate & ifno or boundname & -- & Simulated mass flow rate for a well or group of wells. \\
MWT & fw-rate & ifno or boundname & -- & Simulated mass flow rate for a flowing well or group of flowing wells. \\
MWT & rate-to-mvr & well or boundname & -- & Simulated mass flow rate that is sent to the MVT Package for a well or group of wells.\\
MWT & fw-rate-to-mvr & well or boundname & -- & Simulated mass flow rate that is sent to the MVT Package from a flowing well or group of flowing wells. \\

\label{table:gwt-mwtobstype}
\end{longtable}

\vspace{5mm}
\subsubsection{Example Observation Input File}
\lstinputlisting[style=inputfile]{./mf6ivar/examples/gwt-mwt-example-obs.dat}




\newpage
\subsection{Unsaturated Zone Transport (UZT) Package}
Unsaturated Zone Transport (UZT) Package information is read from the file that is specified by ``UZT6'' as the file type.  There can be as many UZT Packages as necessary for a GWT model. Each UZT Package is designed to work with flows from a corresponding GWF UZF Package. By default \mf uses the UZT package name to determine which UZF Package corresponds to the UZT Package.  Therefore, the package name of the UZT Package (as specified in the GWT name file) must match with the name of the corresponding UZF Package (as specified in the GWF name file).  Alternatively, the name of the flow package can be specified using the FLOW\_PACKAGE\_NAME keyword in the options block.  The GWT UZT Package cannot be used without a corresponding GWF UZF Package.

The UZT Package does not have a dimensions block; instead, dimensions for the UZT Package are set using the dimensions from the corresponding UZF Package.  For example, the UZF Package requires specification of the number of cells (NUZFCELLS).  UZT sets the number of UZT cells equal to NUZFCELLS.  Therefore, the PACKAGEDATA block below must have NUZFCELLS entries in it.

\vspace{5mm}
\subsubsection{Structure of Blocks}
\lstinputlisting[style=blockdefinition]{./mf6ivar/tex/gwt-uzt-options.dat}
\lstinputlisting[style=blockdefinition]{./mf6ivar/tex/gwt-uzt-packagedata.dat}
\lstinputlisting[style=blockdefinition]{./mf6ivar/tex/gwt-uzt-period.dat}

\vspace{5mm}
\subsubsection{Explanation of Variables}
\begin{description}
% DO NOT MODIFY THIS FILE DIRECTLY.  IT IS CREATED BY mf6ivar.py 

\item \textbf{Block: OPTIONS}

\begin{description}
\item \texttt{flow\_package\_name}---keyword to specify the name of the corresponding flow package.  If not specified, then the corresponding flow package must have the same name as this advanced transport package (the name associated with this package in the GWT name file).

\item \texttt{auxiliary}---defines an array of one or more auxiliary variable names.  There is no limit on the number of auxiliary variables that can be provided on this line; however, lists of information provided in subsequent blocks must have a column of data for each auxiliary variable name defined here.   The number of auxiliary variables detected on this line determines the value for naux.  Comments cannot be provided anywhere on this line as they will be interpreted as auxiliary variable names.  Auxiliary variables may not be used by the package, but they will be available for use by other parts of the program.  The program will terminate with an error if auxiliary variables are specified on more than one line in the options block.

\item \texttt{flow\_package\_auxiliary\_name}---keyword to specify the name of an auxiliary variable in the corresponding flow package.  If specified, then the simulated concentrations from this advanced transport package will be copied into the auxiliary variable specified with this name.  Note that the flow package must have an auxiliary variable with this name or the program will terminate with an error.  If the flows for this advanced transport package are read from a file, then this option will have no effect.

\item \texttt{BOUNDNAMES}---keyword to indicate that boundary names may be provided with the list of unsaturated zone flow cells.

\item \texttt{PRINT\_INPUT}---keyword to indicate that the list of unsaturated zone flow information will be written to the listing file immediately after it is read.

\item \texttt{PRINT\_CONCENTRATION}---keyword to indicate that the list of UZF cell concentration will be printed to the listing file for every stress period in which ``CONCENTRATION PRINT'' is specified in Output Control.  If there is no Output Control option and PRINT\_CONCENTRATION is specified, then concentration are printed for the last time step of each stress period.

\item \texttt{PRINT\_FLOWS}---keyword to indicate that the list of unsaturated zone flow rates will be printed to the listing file for every stress period time step in which ``BUDGET PRINT'' is specified in Output Control.  If there is no Output Control option and ``PRINT\_FLOWS'' is specified, then flow rates are printed for the last time step of each stress period.

\item \texttt{SAVE\_FLOWS}---keyword to indicate that unsaturated zone flow terms will be written to the file specified with ``BUDGET FILEOUT'' in Output Control.

\item \texttt{CONCENTRATION}---keyword to specify that record corresponds to concentration.

\item \texttt{concfile}---name of the binary output file to write concentration information.

\item \texttt{BUDGET}---keyword to specify that record corresponds to the budget.

\item \texttt{FILEOUT}---keyword to specify that an output filename is expected next.

\item \texttt{budgetfile}---name of the binary output file to write budget information.

\item \texttt{BUDGETCSV}---keyword to specify that record corresponds to the budget CSV.

\item \texttt{budgetcsvfile}---name of the comma-separated value (CSV) output file to write budget summary information.  A budget summary record will be written to this file for each time step of the simulation.

\item \texttt{TS6}---keyword to specify that record corresponds to a time-series file.

\item \texttt{FILEIN}---keyword to specify that an input filename is expected next.

\item \texttt{ts6\_filename}---defines a time-series file defining time series that can be used to assign time-varying values. See the ``Time-Variable Input'' section for instructions on using the time-series capability.

\item \texttt{OBS6}---keyword to specify that record corresponds to an observations file.

\item \texttt{obs6\_filename}---name of input file to define observations for the UZT package. See the ``Observation utility'' section for instructions for preparing observation input files. Tables \ref{table:gwf-obstypetable} and \ref{table:gwt-obstypetable} lists observation type(s) supported by the UZT package.

\end{description}
\item \textbf{Block: PACKAGEDATA}

\begin{description}
\item \texttt{ifno}---integer value that defines the feature (UZF object) number associated with the specified PACKAGEDATA data on the line. IFNO must be greater than zero and less than or equal to NUZFCELLS. Unsaturated zone flow information must be specified for every UZF cell or the program will terminate with an error.  The program will also terminate with an error if information for a UZF cell is specified more than once.

\item \texttt{strt}---real value that defines the starting concentration for the unsaturated zone flow cell.

\item \textcolor{blue}{\texttt{aux}---represents the values of the auxiliary variables for each unsaturated zone flow. The values of auxiliary variables must be present for each unsaturated zone flow. The values must be specified in the order of the auxiliary variables specified in the OPTIONS block.  If the package supports time series and the Options block includes a TIMESERIESFILE entry (see the ``Time-Variable Input'' section), values can be obtained from a time series by entering the time-series name in place of a numeric value.}

\item \texttt{boundname}---name of the unsaturated zone flow cell.  BOUNDNAME is an ASCII character variable that can contain as many as 40 characters.  If BOUNDNAME contains spaces in it, then the entire name must be enclosed within single quotes.

\end{description}
\item \textbf{Block: PERIOD}

\begin{description}
\item \texttt{iper}---integer value specifying the starting stress period number for which the data specified in the PERIOD block apply.  IPER must be less than or equal to NPER in the TDIS Package and greater than zero.  The IPER value assigned to a stress period block must be greater than the IPER value assigned for the previous PERIOD block.  The information specified in the PERIOD block will continue to apply for all subsequent stress periods, unless the program encounters another PERIOD block.

\item \texttt{ifno}---integer value that defines the feature (UZF object) number associated with the specified PERIOD data on the line. IFNO must be greater than zero and less than or equal to NUZFCELLS.

\item \texttt{uztsetting}---line of information that is parsed into a keyword and values.  Keyword values that can be used to start the UZTSETTING string include: STATUS, CONCENTRATION, INFILTRATION, UZET, and AUXILIARY.  These settings are used to assign the concentration of associated with the corresponding flow terms.  Concentrations cannot be specified for all flow terms.

\begin{lstlisting}[style=blockdefinition]
STATUS <status>
CONCENTRATION <@concentration@>
INFILTRATION <@infiltration@>
UZET <@uzet@>
AUXILIARY <auxname> <@auxval@> 
\end{lstlisting}

\item \texttt{status}---keyword option to define UZF cell status.  STATUS can be ACTIVE, INACTIVE, or CONSTANT. By default, STATUS is ACTIVE, which means that concentration will be calculated for the UZF cell.  If a UZF cell is inactive, then there will be no solute mass fluxes into or out of the UZF cell and the inactive value will be written for the UZF cell concentration.  If a UZF cell is constant, then the concentration for the UZF cell will be fixed at the user specified value.

\item \textcolor{blue}{\texttt{concentration}---real or character value that defines the concentration for the unsaturated zone flow cell. The specified CONCENTRATION is only applied if the unsaturated zone flow cell is a constant concentration cell. If the Options block includes a TIMESERIESFILE entry (see the ``Time-Variable Input'' section), values can be obtained from a time series by entering the time-series name in place of a numeric value.}

\item \textcolor{blue}{\texttt{infiltration}---real or character value that defines the infiltration solute concentration $(ML^{-3})$ for the UZF cell. If the Options block includes a TIMESERIESFILE entry (see the ``Time-Variable Input'' section), values can be obtained from a time series by entering the time-series name in place of a numeric value.}

\item \textcolor{blue}{\texttt{uzet}---real or character value that defines the concentration of unsaturated zone evapotranspiration water $(ML^{-3})$ for the UZF cell. If this concentration value is larger than the simulated concentration in the UZF cell, then the unsaturated zone ET water will be removed at the same concentration as the UZF cell.  If the Options block includes a TIMESERIESFILE entry (see the ``Time-Variable Input'' section), values can be obtained from a time series by entering the time-series name in place of a numeric value.}

\item \texttt{AUXILIARY}---keyword for specifying auxiliary variable.

\item \texttt{auxname}---name for the auxiliary variable to be assigned AUXVAL.  AUXNAME must match one of the auxiliary variable names defined in the OPTIONS block. If AUXNAME does not match one of the auxiliary variable names defined in the OPTIONS block the data are ignored.

\item \textcolor{blue}{\texttt{auxval}---value for the auxiliary variable. If the Options block includes a TIMESERIESFILE entry (see the ``Time-Variable Input'' section), values can be obtained from a time series by entering the time-series name in place of a numeric value.}

\end{description}


\end{description}

\vspace{5mm}
\subsubsection{Example Input File}
\lstinputlisting[style=inputfile]{./mf6ivar/examples/gwt-uzt-example.dat}

\vspace{5mm}
\subsubsection{Available observation types}
Unsaturated Zone Transport Package observations include UZF cell concentration and all of the terms that contribute to the continuity equation for each UZF cell. Additional UZT Package observations include mass flow rates for individual UZF cells, or groups of UZF cells. The data required for each UZT Package observation type is defined in table~\ref{table:gwt-uztobstype}. Negative and positive values for \texttt{uzt} observations represent a loss from and gain to the GWT model, respectively. For all other flow terms, negative and positive values represent a loss from and gain from the UZT package, respectively.

\begin{longtable}{p{2cm} p{2.75cm} p{2cm} p{1.25cm} p{7cm}}
\caption{Available UZT Package observation types} \tabularnewline

\hline
\hline
\textbf{Stress Package} & \textbf{Observation type} & \textbf{ID} & \textbf{ID2} & \textbf{Description} \\
\hline
\endfirsthead

\captionsetup{textformat=simple}
\caption*{\textbf{Table \arabic{table}.}{\quad}Available UZT Package observation types.---Continued} \tabularnewline

\hline
\hline
\textbf{Stress Package} & \textbf{Observation type} & \textbf{ID} & \textbf{ID2} & \textbf{Description} \\
\hline
\endhead


\hline
\endfoot

% general APT observations
UZT & concentration & ifno or boundname & -- & uzt cell concentration. If boundname is specified, boundname must be unique for each uzt cell. \\
UZT & flow-ja-face & ifno or boundname & ifno or -- & Mass flow between two uzt cells.  If a boundname is specified for ID1, then the result is the total mass flow for all uzt cells. If a boundname is specified for ID1 then ID2 is not used.\\
UZT & storage & ifno or boundname & -- & Simulated mass storage flow rate for a uzt cell or group of uzt cells. \\
UZT & constant & ifno or boundname & -- & Simulated mass constant-flow rate for a uzt cell or a group of uzt cells. \\
UZT & from-mvr & ifno or boundname & -- & Simulated mass inflow into a uzt cell or group of uzt cells from the MVT package. Mass inflow is calculated as the product of provider concentration and the mover flow rate. \\
UZT & uzt & ifno or boundname & -- & Mass flow rate for a uzt cell or group of uzt cells and its aquifer connection(s). \\

%observations specific to the uzt package
% infiltration rej-inf uzet rej-inf-to-mvr
UZT & infiltration & ifno or boundname & -- & Infiltration rate applied to a uzt cell or group of uzt cells multiplied by the infiltration concentration. \\
UZT & rej-inf & ifno or boundname & -- & Rejected infiltration rate applied to a uzt cell or group of uzt cells multiplied by the infiltration concentration. \\
UZT & uzet & ifno or boundname & -- & Unsaturated zone evapotranspiration rate applied to a uzt cell or group of uzt cells multiplied by the uzt cell concentration. \\
UZT & rej-inf-to-mvr & ifno or boundname & -- & Rejected infiltration rate applied to a uzt cell or group of uzt cells multiplied by the infiltration concentration that is sent to the mover package. \\

\label{table:gwt-uztobstype}
\end{longtable}

\vspace{5mm}
\subsubsection{Example Observation Input File}
\lstinputlisting[style=inputfile]{./mf6ivar/examples/gwt-uzt-example-obs.dat}




\newpage
\subsection{Flow Model Interface (FMI) Package}
Flow Model Interface (FMI) Package information is read from the file that is specified by ``FMI6'' as the file type.  An FMI Package specified by the user is not required if simulated groundwaters flows are available from a corresponding GWF Model that is running during the simulation.  In this case, flows from the GWF Model are made available to the transport model by the GWF-GWT Exchange.  An FMI Package also is not required if one can safely assume that there is no groundwater flow and a corresponding GWF Model does not exist.  In this case, cells are assumed to be fully saturated with stagnant water, and changes in concentration occur solely through molecular diffusion, sorbtion, and decay.  If the objective is to use the results from a previous GWF Model simulation as input to the transport model, then this FMI Package can be used to specify the GWF budget and head files.  The budget file is read during the simulation and must contain all of intercell flows and package flows for every stress period and time step in the simulation.  The binary head file must also contain heads for every stress period and time step in the simulation.  Only one FMI Package can be specified for a GWT model. 

\vspace{5mm}
\subsubsection{Structure of Blocks}
\lstinputlisting[style=blockdefinition]{./mf6ivar/tex/gwt-fmi-options.dat}

\vspace{5mm}
\subsubsection{Explanation of Variables}
\begin{description}
% DO NOT MODIFY THIS FILE DIRECTLY.  IT IS CREATED BY mf6ivar.py 

\item \textbf{Block: OPTIONS}

\begin{description}
\item \texttt{FLOW\_IMBALANCE\_CORRECTION}---correct for an imbalance in flows by assuming that any residual flow error comes in or leaves at the concentration of the cell.

\end{description}
\item \textbf{Block: PACKAGEDATA}

\begin{description}
\item \texttt{flowtype}---is the word GWFBUDGET, GWFHEAD, GWFMOVER or the name of an advanced GWF stress package.  If GWFBUDGET is specified, then the corresponding file must be a budget file from a previous GWF Model run.  If an advanced GWF stress package name appears then the corresponding file must be the budget file saved by a LAK, SFR, MAW or UZF Package.

\item \texttt{FILEIN}---keyword to specify that an input filename is expected next.

\item \texttt{fname}---is the name of the file containing flows.  The path to the file should be included if the file is not located in the folder where the program was run.

\end{description}


\end{description}

\vspace{5mm}
\subsubsection{Example Input File}
\lstinputlisting[style=inputfile]{./mf6ivar/examples/gwt-fmi-example.dat}



%\newpage
%\subsection{Observation (OBS) Utility for a GWT Model}
%GWT & concentration & cellid & -- & Concentration at a specified cell. \\
GWT & flow-ja-face & cellid & cellid & Mass flow in dimensions of mass per time between two adjacent cells.  The mass flow rate includes the contributions from both advection and dispersion if those packages are active

