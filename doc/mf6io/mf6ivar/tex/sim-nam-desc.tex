% DO NOT MODIFY THIS FILE DIRECTLY.  IT IS CREATED BY mf6ivar.py 

\item \textbf{Block: OPTIONS}

\begin{description}
\item \texttt{CONTINUE}---keyword flag to indicate that the simulation should continue even if one or more solutions do not converge.

\item \texttt{NOCHECK}---keyword flag to indicate that the model input check routines should not be called prior to each time step. Checks are performed by default.

\item \texttt{memory\_print\_option}---is a flag that controls printing of detailed memory manager usage to the end of the simulation list file.  NONE means do not print detailed information. SUMMARY means print only the total memory for each simulation component. ALL means print information for each variable stored in the memory manager. NONE is default if MEMORY\_PRINT\_OPTION is not specified.

\item \texttt{maxerrors}---maximum number of errors that will be stored and printed.

\item \texttt{PRINT\_INPUT}---keyword to activate printing of simulation input summaries to the simulation list file (mfsim.lst). With this keyword, input summaries will be written for those packages that support newer input data model routines.  Not all packages are supported yet by the newer input data model routines.

\item \texttt{HPC6}---keyword to specify that record corresponds to a hpc file.

\item \texttt{FILEIN}---keyword to specify that an input filename is expected next.

\item \texttt{hpc6\_filename}---name of input file to define HPC file settings for the HPC package. See the ``HPC File'' section for instructions for preparing HPC input files.

\end{description}
\item \textbf{Block: TIMING}

\begin{description}
\item \texttt{tdis6}---is the name of the Temporal Discretization (TDIS) Input File.

\end{description}
\item \textbf{Block: MODELS}

\begin{description}
\item \texttt{mtype}---is the type of model to add to simulation.

\item \texttt{mfname}---is the file name of the model name file.

\item \texttt{mname}---is the user-assigned name of the model.  The model name cannot exceed 16 characters and must not have blanks within the name.  The model name is case insensitive; any lowercase letters are converted and stored as upper case letters.

\end{description}
\item \textbf{Block: EXCHANGES}

\begin{description}
\item \texttt{exgtype}---is the exchange type.

\item \texttt{exgfile}---is the input file for the exchange.

\item \texttt{exgmnamea}---is the name of the first model that is part of this exchange.

\item \texttt{exgmnameb}---is the name of the second model that is part of this exchange.

\end{description}
\item \textbf{Block: SOLUTIONGROUP}

\begin{description}
\item \texttt{group\_num}---is the group number of the solution group.  Solution groups must be numbered sequentially, starting with group number one.

\item \texttt{mxiter}---is the maximum number of outer iterations for this solution group.  The default value is 1.  If there is only one solution in the solution group, then MXITER must be 1.

\item \texttt{slntype}---is the type of solution.  The Integrated Model Solution (IMS6) is the only supported option in this version.

\item \texttt{slnfname}---name of file containing solution input.

\item \texttt{slnmnames}---is the array of model names to add to this solution.  The number of model names is determined by the number of model names the user provides on this line.

\end{description}

