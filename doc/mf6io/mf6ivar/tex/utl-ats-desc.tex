% DO NOT MODIFY THIS FILE DIRECTLY.  IT IS CREATED BY mf6ivar.py 

\item \textbf{Block: DIMENSIONS}

\begin{description}
\item \texttt{maxats}---is the number of records in the subsequent perioddata block that will be used for adaptive time stepping.

\end{description}
\item \textbf{Block: PERIODDATA}

\begin{description}
\item \texttt{iperats}---is the period number to designate for adaptive time stepping.  The remaining ATS values on this line will apply to period iperats.

\item \texttt{dt0}---is the initial time step length for period iperats.  If the value is zero, then the final step from the previous stress period will be used as the initial time step.

\item \texttt{dtmin}---is the minimum time step length for this period.  This value must be less than or equal to tsmax and less than or equal to dt0.

\item \texttt{dtmax}---is the maximum time step length for this period.  This value must be greater than or equal to tsmin and greater than or equal to dt0.

\end{description}

