\subsection{\mf Extended Build}

Next to the standard \mf executable, a second, extended version of the program is made available. This version comes with more advanced functionality for which it partially relies on third-party libraries. Currently this concerns the parallel computing capability and the use of NetCDF4 for I/O data. Because the external dependencies increase the complexity of the installation procedure, \mf will remain available in its core set of functionality.

Extended \mf contains all features available in the standard edition, runs on the same input configuration, and produces the same results. Reversely, when running with the standard executable, some features described in this document (HPC Utility, NetCDF4 I/O) will not be available and their configuration will be ignored or an error is reported. These features will be labeled accordingly below.

\subsection{NetCDF Export Files}

The extended build of \mf can optionally create model NetCDF export files.  The files are NetCDF-4 format, and as such support some HDF5 features such as per-variable compression.  \mf supports two types of NetCDF exports, referred to here as ''structured'' and ''mesh''.

\mf NetCDF exports by default contain model dependent variable timeseries data, e.g. head.  In addition, \mf package griddata input arrays can be configured to be written to the export file.  Creating export files containing only one or the other of these these types of data form two distinct uses for the exports.  Generating exports of these two types will described below.

These capabilities are dependent on the earlier described Input Data Processor (IDP).  In particular, the package array export capability described below is limited to packages currently supported by IDP.

For information about the installation and capabilities of the extended build of \mf please refer to \url{https://github.com/MODFLOW-USGS/modflow6/wiki}.

\subsubsection{Mesh Exports}
\mf mesh exports comply with UGRID 1.0 conventions and are based on the UGRID 3D layered mesh topology.  A UGRID based NetCDF file explicitly describes the grid with a set of variables that gridded data can be associated with.  The layered mesh topology describes the mesh as a 2D layer and is based on the assumption that every layer shares the same grid specification.  All associated 3D gridded data variables are split into layer variables whose data reflects a mapping to locations on faces, or cells, described by the mesh.  \mf mesh exports can be generated with a DIS or DISV based GWF, GWT, or GWE model.

\subsubsection{Structured Exports}
\mf structured NetCDF exports take the approach of defining x and y geometric coordinate variables.  Gridded data variables specify a mapping either directly to the coordinate variables or to latitude and longitude arrays that are themselves associated with the coordinates.  The x and y coordinates define cell centers on a grid layer and as such the grid is not explicitly described.  Visualizing tools typically display a ''data mesh'' as grid nodes aren't described in the dataset. \mf structured exports can be generated with a DIS based GWF, GWT, or GWE model.

\subsubsection{Dependent variable timeseries output}
A typical use case creates an export that capture timeseries output for the model dependent variable.  This export can be optioned simply by adding the ''EXPORT\_NETCDF'' keyword to a GWF, GWT, or GWE model name file.  This keyword takes an argument for the NetCDF export type, either ''STRUCTURED'' or ''MESH'' (or alternatively, ''UGRID'').  Configured in this way, the export will contain only the mesh or x and y coordinate variables and the dependent variable(s).  Additional configuration options are possible when a NetCDF configuration package (UTL-NCF) is created, such as per-variable compression, chunking options, and options meant to support visualization in post-processing tools like QGIS.  This package is described below.

\subsubsection{Exporting array input}
Packages that support exporting griddata input arrays to model NetCDF files support the ''EXPORT\_ARRAY\_NETCDF'' keyword.  Using this keyword, when model NetCDF export also is active, has the effect of writing all package griddata arrays to the NetCDF export.  Writing candidate arrays to a NetCDF file offers an alternative to storing them in an ascii or binary input file.

One approach to converting existing ascii or binary inputs to NetCDF inputs is described next.  \mf supports a validate mode option intended to support error checking.  In this mode no matrix equations are assembled or solved and no solution based outputs are created.  Input, however, is still read in validate mode and thus can be written to NetCDF exports when configured. Specific IDP supported packages can be configured to export griddata array input by using the 'EXPORT\_ARRAY\_NETCDF'' package option when running the simulation in validate mode.  When configured in this way, \mf will generate a NetCDF export file, absent a dependent or time coordinate variable, that can serve as a separate input file for model gridded package data.

\subsubsection{NetCDF as simulation input}
\mf can read NetCDF files as model inputs containing package griddata array input. Files generated using the validate method in the previous section are suitable as inputs- note that \mf expects specific annotations for NetCDF input variables and that these are written by default into \mf NetCDF exports. An example follows showing a model name file that configures a NetCDF input and IC and NPF input files that specify specific griddata parameters should be read from the model NetCDF input:
\lstinputlisting[style=inputfile]{./mf6ivar/examples/netcdf-nam-example.dat}
\lstinputlisting[style=inputfile]{./mf6ivar/examples/netcdf-ic-example.dat}
\lstinputlisting[style=inputfile]{./mf6ivar/examples/netcdf-npf-example.dat}
The above name file configures a NetCDF model input file, using the ”NETCDF FILEIN” keyword set, which contains IC and NPF package griddata arrays. The model IC and NPF inputs below the name file indicate that the STRT, ICELLTYPE and K parameters should be read from the configured model NetCDF input by using the ”NETCDF” keyword after each griddata block parameter name. Note also in this example that EXPORT\_NETCDF is configured, which will generate a timeseries export of the model dependent variable over the course of the simulation run.


%\SECTION{High Performance Computing (HPC) Utility -- \textcolor{red}{Extended MODFLOW only}}
\subsection{High Performance Computing (HPC) Utility -- \textcolor{red}{Extended MODFLOW only}}
The High Performance Computing (HPC) utility file for the simulation can be activated by specifying the HPC6 option in the simulation name file.  It's main purpose is to assign the models in a parallel simulation to the available CPU cores for cases where the internal distribution algorithm is not satisfactory. If activated, \mf will read HPC input according to the following description.

\vspace{5mm}
\subsubsection{Structure of Blocks}
\lstinputlisting[style=blockdefinition]{./mf6ivar/tex/utl-hpc-options.dat}
\lstinputlisting[style=blockdefinition]{./mf6ivar/tex/utl-hpc-partitions.dat}

\vspace{5mm}
\subsubsection{Explanation of Variables}
\begin{description}
% DO NOT MODIFY THIS FILE DIRECTLY.  IT IS CREATED BY mf6ivar.py 

\item \textbf{Block: OPTIONS}

\begin{description}
\end{description}
\item \textbf{Block: PARTITIONS}

\begin{description}
\item \texttt{mname}---is the unique model name.

\item \texttt{mrank}---is the zero-based partition number (also: MPI rank or processor id) to which the model will be assigned.

\end{description}


\end{description}

\vspace{5mm}
\subsubsection{Example Input File}
Example 1: HPC input file distributing 6 models over 4 available CPU cores.
\lstinputlisting[style=inputfile]{./mf6ivar/examples/utl-hpc-example1.dat}

\vspace{5mm}
Example 2: HPC input file distributing 3 GWF models coupled individually to 3 GWT models over 2 available CPU cores. Note that the GWT models have to be assigned the same partition numbers as their GWF counterparts.
\lstinputlisting[style=inputfile]{./mf6ivar/examples/utl-hpc-example2.dat}

%\SECTION{NetCDF configuration (NCF) Utility -- \textcolor{red}{Extended MODFLOW only}}
\subsection{NetCDF configuration (NCF) Utility -- \textcolor{red}{Extended MODFLOW only}}
The NetCDF (NCF) configuration utility can be activated by specifying the NCF6 option in a DIS or DISV input file.

The NCF configuration utility applies to model NetCDF exports that are elected with the EXPORT\_NETCDF keyword in a model name file. The EXPORT\_NETCDF keyword can be used whether or not this configuration package is defined. When defined, this package provides options related to data variable chunking, compression and grid mapping (projections).

\subsubsection{Structure of Blocks}
\lstinputlisting[style=blockdefinition]{./mf6ivar/tex/utl-ncf-options.dat}
\lstinputlisting[style=blockdefinition]{./mf6ivar/tex/utl-ncf-dimensions.dat}
\lstinputlisting[style=blockdefinition]{./mf6ivar/tex/utl-ncf-griddata.dat}
\vspace{5mm}

\subsubsection{Explanation of Variables}
\begin{description}
% DO NOT MODIFY THIS FILE DIRECTLY.  IT IS CREATED BY mf6ivar.py 

\item \textbf{Block: OPTIONS}

\begin{description}
\item \texttt{ogc\_wkt}---is the CRS well-known text (WKT) string.

\item \texttt{deflate}---is the deflate level (0=min, 9=max) for per variable compression in the netcdf file. Defining the parameter activates per variable compression in the export file at the level specified.

\item \texttt{SHUFFLE}---is the keyword used to turn on the netcdf variable shuffle filter when the deflate option is also set. The shuffle filter has the effect of storing the first byte of all of a variable's values in a chunk contiguously, followed by all the second bytes, etc. This can be an optimization for compression with certain types of data.

\item \texttt{CHUNKING\_UGRID}---is a keyword for providing ugrid dimension chunk sizes. Chunking can dramatically impact data access times and is highly dependent on access patterns (timeseries vs spatial, for example). It can also significanlty impact compressibility of the data.

\item \texttt{ugc\_time}---is the keyword used to provide a ugrid time dimension chunk size.

\item \texttt{ugc\_face}---is the keyword used to provide a ugrid face dimension chunk size.

\end{description}


\end{description}
\vspace{5mm}

\subsubsection{Example Input File}
\lstinputlisting[style=inputfile]{./mf6ivar/examples/utl-ncf-example.dat}

