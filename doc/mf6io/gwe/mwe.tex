Multi-Aquifer Well Energy Transport (MWE) Package information is read from the file that is specified by ``MWE6'' as the file type.  There can be as many MWE Packages as necessary for a GWE model. Each MWE Package is designed to work with flows from a corresponding GWF MAW Package. By default \mf uses the MWE package name to determine which MAW Package corresponds to the MWE Package.  Therefore, the package name of the MWE Package (as specified in the GWE name file) must match with the name of the corresponding MAW Package (as specified in the GWF name file).  Alternatively, the name of the flow package can be specified using the FLOW\_PACKAGE\_NAME keyword in the options block.  The GWE MWE Package cannot be used without a corresponding GWF MAW Package.

The MWE Package does not have a dimensions block; instead, dimensions for the MWE Package are set using the dimensions from the corresponding MAW Package.  For example, the MAW Package requires specification of the number of wells (NMAWWELLS).  MWE sets the number of wells equal to NMAWWELLS.  Therefore, the PACKAGEDATA block below must have NMAWWELLS entries in it.

\vspace{5mm}
\subsubsection{Structure of Blocks}
\lstinputlisting[style=blockdefinition]{./mf6ivar/tex/gwe-mwe-options.dat}
\lstinputlisting[style=blockdefinition]{./mf6ivar/tex/gwe-mwe-packagedata.dat}
\lstinputlisting[style=blockdefinition]{./mf6ivar/tex/gwe-mwe-period.dat}

\vspace{5mm}
\subsubsection{Explanation of Variables}
\begin{description}
\input{./mf6ivar/tex/gwe-mwe-desc.tex}
\end{description}

\vspace{5mm}
\subsubsection{Example Input File}
\lstinputlisting[style=inputfile]{./mf6ivar/examples/gwe-mwe-example.dat}

\vspace{5mm}
\subsubsection{Available observation types}
Multi-Aquifer Well Energy Transport Package observations include well temperature and all of the terms that contribute to the continuity equation for each well. Additional MWE Package observations include energy flow rates for individual wells, or groups of wells; the well volume (\texttt{volume}); and the conductance for a well-aquifer connection conductance (\texttt{conductance}). The data required for each MWE Package observation type is defined in table~\ref{table:gwe-mweobstype}. Negative and positive values for \texttt{mwe} observations represent a loss from and gain to the GWE model, respectively. For all other flow terms, negative and positive values represent a loss from and gain from the MWE package, respectively.

\begin{longtable}{p{2cm} p{2.75cm} p{2cm} p{1.25cm} p{7cm}}
\caption{Available MWE Package observation types} \tabularnewline

\hline
\hline
\textbf{Stress Package} & \textbf{Observation type} & \textbf{ID} & \textbf{ID2} & \textbf{Description} \\
\hline
\endfirsthead

\captionsetup{textformat=simple}
\caption*{\textbf{Table \arabic{table}.}{\quad}Available MWE Package observation types.---Continued} \tabularnewline

\hline
\hline
\textbf{Stress Package} & \textbf{Observation type} & \textbf{ID} & \textbf{ID2} & \textbf{Description} \\
\hline
\endhead


\hline
\endfoot

% general APT observations
MWE & temperature & mawno or boundname & -- & Well temperature. If boundname is specified, boundname must be unique for each well. \\
%flowjaface not included
MWE & storage & mawno or boundname & -- & Simulated energy storage flow rate for a well or group of wells. \\
MWE & constant & mawno or boundname & -- & Simulated energy constant-flow rate for a well or group of wells. \\
MWE & from-mvr & mawno or boundname & -- & Simulated energy inflow into a well or group of wells from the MVE package. Energy inflow is calculated as the product of provider temperature and the mover flow rate. \\
MWE & mwe & mawno or boundname & \texttt{iconn} or -- & Energy flow rate for a well or group of wells and its aquifer connection(s). If boundname is not specified for ID, then the simulated well-aquifer flow rate at a specific well connection is observed. In this case, ID2 must be specified and is the connection number \texttt{iconn} for well \texttt{mawno}. \\

% observations specific to the mwe package
MWE & rate & mawno or boundname & -- & Simulated energy flow rate for a well or group of wells. \\
MWE & fw-rate & mawno or boundname & -- & Simulated energy flow rate for a flowing well or group of flowing wells. \\
MWE & rate-to-mvr & well or boundname & -- & Simulated energy flow rate that is sent to the MVE Package for a well or group of wells.\\
MWE & fw-rate-to-mvr & well or boundname & -- & Simulated energy flow rate that is sent to the MVE Package from a flowing well or group of flowing wells. \\

\label{table:gwe-mweobstype}
\end{longtable}

\vspace{5mm}
\subsubsection{Example Observation Input File}
\lstinputlisting[style=inputfile]{./mf6ivar/examples/gwe-mwe-example-obs.dat}


