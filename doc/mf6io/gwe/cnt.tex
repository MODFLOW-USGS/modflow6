Constant Temperature (CNT) Package information is read from the file that is specified by ``CNT6'' as the file type.  Any number of CNT Packages can be specified for a single GWE model, but the same cell cannot be designated as a constant temperature by more than one CNT entry. 

\vspace{5mm}
\subsubsection{Structure of Blocks}
\vspace{5mm}

\noindent \textit{FOR EACH SIMULATION}
\lstinputlisting[style=blockdefinition]{./mf6ivar/tex/gwe-cnt-options.dat}
\lstinputlisting[style=blockdefinition]{./mf6ivar/tex/gwe-cnt-dimensions.dat}
\vspace{5mm}
\noindent \textit{FOR ANY STRESS PERIOD}
\lstinputlisting[style=blockdefinition]{./mf6ivar/tex/gwe-cnt-period.dat}
\packageperioddescription

\vspace{5mm}
\subsubsection{Explanation of Variables}
\begin{description}
\input{./mf6ivar/tex/gwe-cnt-desc.tex}
\end{description}

\vspace{5mm}
\subsubsection{Example Input File}
\lstinputlisting[style=inputfile]{./mf6ivar/examples/gwe-cnt-example.dat}

\vspace{5mm}
\subsubsection{Available observation types}
CNT Package observations are limited to the simulated constant temperature energy flow rate (\texttt{cnt}). The data required for the CNT Package observation type is defined in table~\ref{table:gwe-cntobstype}. Negative and positive values for an observation represent a loss from and gain to the GWE model, respectively.

\begin{longtable}{p{2cm} p{2.75cm} p{2cm} p{1.25cm} p{7cm}}
\caption{Available CNT Package observation types} \tabularnewline

\hline
\hline
\textbf{Model} & \textbf{Observation type} & \textbf{ID} & \textbf{ID2} & \textbf{Description} \\
\hline
\endhead

\hline
\endfoot

\input{../Common/gwe-cntobs.tex}
\label{table:gwe-cntobstype}
\end{longtable}

\vspace{5mm}
\subsubsection{Example Observation Input File}
\lstinputlisting[style=inputfile]{./mf6ivar/examples/gwe-cnt-example-obs.dat}
