The Adaptive Time Step (ATS) utility for the TDIS Package can be activated by specifying the ATS6 option in the TDIS input file.  If activated, \mf will read ATS input according to the following description.

The adaptive time step utility is activated for any stress periods that are listed in the PERIODDATA block below.  If a stress period is adaptive, then the NSTP and TSMULT parameters in the TDIS input file have no affect on time step progression.  Instead the ATS settings specified for the period are used to control the time step progression.

The ATS implementation implemented in \mf is based on the approach implemented in MODFLOW-USG.  There are two fundamental parts to the ATS utility.  The first is the capability to handle failure of a solution to converge.  If ATS is active for a stress period in which the solution fails to converge, then the program will continue to try smaller time steps until convergence is achieved or the length of the time step reaches the lower allowable limit.  Once this lower limit is reached, then the program will either stop and write concluding information, or the program will continue to the next time step if the CONTINUE option is specified in the simulation name file.

The second fundamental part of the ATS utility is dynamic adjustment of the time step size according to simulation behavior.  The ATS implementation is generic in that any model, exchange, or solution can submit a preferred time step length, and ATS will proceed with the smallest submitted time step.

\vspace{5mm}
\subsection{Structure of Blocks}
%\lstinputlisting[style=blockdefinition]{./mf6ivar/tex/sim-tdis-options.dat}
\lstinputlisting[style=blockdefinition]{./mf6ivar/tex/utl-ats-dimensions.dat}
\lstinputlisting[style=blockdefinition]{./mf6ivar/tex/utl-ats-perioddata.dat}

\vspace{5mm}
\subsection{Explanation of Variables}
\begin{description}
% DO NOT MODIFY THIS FILE DIRECTLY.  IT IS CREATED BY mf6ivar.py 

\item \textbf{Block: DIMENSIONS}

\begin{description}
\item \texttt{maxats}---is the number of records in the subsequent perioddata block that will be used for adaptive time stepping.

\end{description}
\item \textbf{Block: PERIODDATA}

\begin{description}
\item \texttt{iperats}---is the period number to designate for adaptive time stepping.  The remaining ATS values on this line will apply to period iperats.

\item \texttt{dt0}---is the initial time step length for period iperats.  If the value is zero, then the final step from the previous stress period will be used as the initial time step.

\item \texttt{dtmin}---is the minimum time step length for this period.  This value must be less than or equal to tsmax and less than or equal to dt0.

\item \texttt{dtmax}---is the maximum time step length for this period.  This value must be greater than or equal to tsmin and greater than or equal to dt0.

\end{description}


\end{description}

\vspace{5mm}
\subsection{Example Input File}
\lstinputlisting[style=inputfile]{./mf6ivar/examples/utl-ats-example.dat}

