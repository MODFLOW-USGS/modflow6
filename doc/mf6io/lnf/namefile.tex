The LNF Model Name File specifies the options and packages that are active for a GWF model.  The Name File contains two blocks: OPTIONS  and PACKAGES. The length of each line must be 299 characters or less. The lines in each block can be in any order.  Files listed in the PACKAGES block must exist when the program starts.

Comment lines are indicated when the first character in a line is one of the valid comment characters.  Commented lines can be located anywhere in the file. Any text characters can follow the comment character. Comment lines have no effect on the simulation; their purpose is to allow users to provide documentation about a particular simulation.

\vspace{5mm}
\subsubsection{Structure of Blocks}
\lstinputlisting[style=blockdefinition]{./mf6ivar/tex/gwf-nam-options.dat}
\lstinputlisting[style=blockdefinition]{./mf6ivar/tex/gwf-nam-packages.dat}

\vspace{5mm}
\subsubsection{Explanation of Variables}
\begin{description}
\input{./mf6ivar/tex/lnf-nam-desc.tex}
\end{description}

\begin{table}[H]
\caption{Ftype values described in this report.  The \texttt{Pname} column indicates whether or not a package name can be provided in the name file}
\small
\begin{center}
\begin{tabular*}{\columnwidth}{l l l}
\hline
\hline
Ftype & Input File Description & \texttt{Pname}\\
\hline
DISL6 & Discretization by Vertices Input File \\
IC6 & Initial Conditions Package \\
OC6 & Output Control Option \\
\hline
\end{tabular*}
\label{table:ftype}
\end{center}
\normalsize
\end{table}

\vspace{5mm}
\subsubsection{Example Input File}
\lstinputlisting[style=inputfile]{./mf6ivar/examples/lnf-nam-example.dat}

