Discretization information for DISL grids is read from the file that is specified by ``DISL6'' as the file type. The approach for numbering cell and cell vertices for the DISL Package is shown in figure~\ref{fig:gwf-fig3-2}--REPLACE FIG.  The list of vertices for a cell must be in ordered in either an upstream or downstream direction. 

\begin{figure}[ht]
	\centering
	\includegraphics[scale=1.0]{Figures/gwf-fig3-2}
	\caption{Schematic diagram showing the vertices and cells defined using the Discretization by Vertices Package. The list of vertices used to define each cell must be in clockwise order.  From \cite{modflow6gwf}}
	\label{fig:gwf-fig3-2}
\end{figure}


\vspace{5mm}
\subsubsection{Structure of Blocks}
\lstinputlisting[style=blockdefinition]{./mf6ivar/tex/lnf-disl-options.dat}
\lstinputlisting[style=blockdefinition]{./mf6ivar/tex/lnf-disl-dimensions.dat}
\lstinputlisting[style=blockdefinition]{./mf6ivar/tex/lnf-disl-griddata.dat}
\lstinputlisting[style=blockdefinition]{./mf6ivar/tex/lnf-disl-vertices.dat}
\lstinputlisting[style=blockdefinition]{./mf6ivar/tex/lnf-disl-cell1d.dat}

\vspace{5mm}
\subsubsection{Explanation of Variables}
\begin{description}
\input{./mf6ivar/tex/lnf-disl-desc.tex}
\end{description}

\vspace{5mm}
\subsubsection{Example Input File}
\lstinputlisting[style=inputfile]{./mf6ivar/examples/lnf-disl-example.dat}
