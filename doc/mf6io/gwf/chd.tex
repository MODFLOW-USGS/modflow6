
Input to the Constant-Head (CHD) Package is read from the file that has type ``CHD6'' in the Name File.  Any number of CHD Packages can be specified for a single groundwater flow model; however, an error will occur if a CHD Package attempts to make a GWF cell a constant-head cell when that cell has already been designated as a constant-head cell either within the present CHD Package or within another CHD Package.

In previous MODFLOW versions, it was not possible to convert a constant-head cell to an active cell.  Once a cell was designated as a constant-head cell, it remained a constant-head cell until the end of the end of the simulation.  In \mf a constant-head cell will become active again if it is not included as a constant-head cell in subsequent stress periods.

Previous MODFLOW versions allowed specification of SHEAD and EHEAD, which were the starting and ending prescribed heads for a stress period.  Linear interpolation was used to calculate a value for each time step.  In \mf only a single head value can be specified for any constant-head cell in any stress period.  The time-series functionality must be used in order to interpolate values to individual time steps.  

\vspace{5mm}
\subsubsection{Structure of Blocks}
\vspace{5mm}

\noindent \textit{FOR EACH SIMULATION}
\lstinputlisting[style=blockdefinition]{./mf6ivar/tex/gwf-chd-options.dat}
\lstinputlisting[style=blockdefinition]{./mf6ivar/tex/gwf-chd-dimensions.dat}
\vspace{5mm}
\noindent \textit{FOR ANY STRESS PERIOD}
\lstinputlisting[style=blockdefinition]{./mf6ivar/tex/gwf-chd-period.dat}
\packageperioddescription

\vspace{5mm}
\subsubsection{Explanation of Variables}
\begin{description}
\input{./mf6ivar/tex/gwf-chd-desc.tex}
\end{description}

\vspace{5mm}
\subsubsection{Example Input File}
\lstinputlisting[style=inputfile]{./mf6ivar/examples/gwf-chd-example.dat}

\vspace{5mm}
\subsubsection{Available observation types}
CHD Package observations are limited to the simulated constant head flow rate (\texttt{chd}). The data required for the CHD Package observation type is defined in table~\ref{table:gwf-chdobstype}. Negative and positive values for an observation represent a loss from and gain to the GWF model, respectively.

\begin{longtable}{p{2cm} p{2.75cm} p{2cm} p{1.25cm} p{7cm}}
\caption{Available CHD Package observation types} \tabularnewline

\hline
\hline
\textbf{Model} & \textbf{Observation type} & \textbf{ID} & \textbf{ID2} & \textbf{Description} \\
\hline
\endhead

\hline
\endfoot

CHD & chd & cellid or boundname & -- & Flow between the groundwater system and a constant-head boundary or a group of cells with constant-head boundaries.

\label{table:gwf-chdobstype}
\end{longtable}

\vspace{5mm}
\subsubsection{Example Observation Input File}
\lstinputlisting[style=inputfile]{./mf6ivar/examples/gwf-chd-example-obs.dat}

