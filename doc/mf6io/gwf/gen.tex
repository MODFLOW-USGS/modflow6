Input to the General Boundary (GEN) Package is read from the file that has type ``GEN6'' in the Name File.  Any number of GEN Packages can be specified for a single groundwater flow model. Stress period data are not specified in the GEN Package input file but are provided to the package using the \mff Application Programming Interface (API).

\vspace{5mm}
\subsubsection{Structure of Blocks}
\vspace{5mm}

\noindent \textit{FOR EACH SIMULATION}
\lstinputlisting[style=blockdefinition]{./mf6ivar/tex/gwf-gen-options.dat}
\lstinputlisting[style=blockdefinition]{./mf6ivar/tex/gwf-gen-dimensions.dat}

\vspace{5mm}
\subsubsection{Explanation of Variables}
\begin{description}
\input{./mf6ivar/tex/gwf-gen-desc.tex}
\end{description}

\vspace{5mm}
\subsubsection{Example Input File}
\lstinputlisting[style=inputfile]{./mf6ivar/examples/gwf-gen-example.dat}

\vspace{5mm}
\subsubsection{Available observation types}
General Boundary Package observations include the simulated general boundary flow rates (\texttt{gen}) and the general boundary discharge that is available for the MVR package (\texttt{to-mvr}). The data required for each GEN Package observation type is defined in table~\ref{table:gwf-genobstype}. The sum of \texttt{gen} and \texttt{to-mvr} is equal to the simulated general boundary flow rate. Negative and positive values for an observation represent a loss from and gain to the GWF model, respectively.

\begin{longtable}{p{2cm} p{2.75cm} p{2cm} p{1.25cm} p{7cm}}
\caption{Available GEN Package observation types} \tabularnewline

\hline
\hline
\textbf{Stress Package} & \textbf{Observation type} & \textbf{ID} & \textbf{ID2} & \textbf{Description} \\
\hline
\endhead

\hline
\endfoot

GEN & gen & cellid or boundname & -- & Flow between the groundwater system and a general boundary or group of general boundaries. \\
GEN & to-mvr & cellid or boundname & -- & General boundary discharge that is available for the MVR package from a general boundary or group of general boundaries.

\label{table:gwf-genobstype}
\end{longtable}

\vspace{5mm}
\subsubsection{Example Observation Input File}
\lstinputlisting[style=inputfile]{./mf6ivar/examples/gwf-gen-example-obs.dat}
