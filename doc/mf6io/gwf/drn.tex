Input to the Drain (DRN) Package is read from the file that has type ``DRN6'' in the Name File.  Any number of DRN Packages can be specified for a single groundwater flow model.

\vspace{5mm}
\subsubsection{Structure of Blocks}
\vspace{5mm}

\noindent \textit{FOR EACH SIMULATION}
\lstinputlisting[style=blockdefinition]{./mf6ivar/tex/gwf-drn-options.dat}
\lstinputlisting[style=blockdefinition]{./mf6ivar/tex/gwf-drn-dimensions.dat}
\vspace{5mm}
\noindent \textit{FOR ANY STRESS PERIOD}
\lstinputlisting[style=blockdefinition]{./mf6ivar/tex/gwf-drn-period.dat}
\packageperioddescription

\vspace{5mm}
\subsubsection{Explanation of Variables}
\begin{description}
\input{./mf6ivar/tex/gwf-drn-desc.tex}
\end{description}

\vspace{5mm}
\subsubsection{Example Input File}
\lstinputlisting[style=inputfile]{./mf6ivar/examples/gwf-drn-example.dat}

\vspace{5mm}
\subsubsection{Available observation types}
Drain Package observations include the simulated drain rates (\texttt{drn}) and the drain discharge that is available for the MVR package (\texttt{to-mvr}). The data required for each DRN Package observation type is defined in table~\ref{table:gwf-drnobstype}. The sum of \texttt{drn} and \texttt{to-mvr} is equal to the simulated drain discharge rate for a drain boundary or group of drain boundaries.

\begin{longtable}{p{2cm} p{2.75cm} p{2cm} p{1.25cm} p{7cm}}
\caption{Available DRN Package observation types} \tabularnewline

\hline
\hline
\textbf{Model} & \textbf{Observation type} & \textbf{ID} & \textbf{ID2} & \textbf{Description} \\
\hline
\endhead

\hline
\endfoot

DRN & drn & cellid or boundname & -- & Flow between the groundwater system and a drain boundary or group of drain boundaries. \\
DRN & to-mvr & cellid or boundname & -- & Drain boundary discharge that is available for the MVR package for a drain boundary or a group of drain boundaries.
\label{table:gwf-drnobstype}
\end{longtable}

\vspace{5mm}
\subsubsection{Example Observation Input File}
\lstinputlisting[style=inputfile]{./mf6ivar/examples/gwf-drn-example-obs.dat}
