% Use this template for starting initializing the release notes
% after a release has just been made.
	
	\item \currentmodflowversion
	
	\underline{NEW FUNCTIONALITY}
	\begin{itemize}
		\item A new Groundwater Energy (GWE) transport model is introduced to the code base for simulating heat transport in the subsurface.  Additional information for activating the GWE model type within a MODFLOW 6 simulation is available within the mf6io.pdf document.  New example problems have been developed for testing and demonstrating GWE capabilities (in addition to other internal tests that help verify the accuracy of GWE); however, additional changes to the code and input may be necessary in response to user needs and further testing.
	    \item A new capability has been introduced to create parameter layer export files of user input data for packages including DIS, DISV, IC, NPF, DSP(GWT), MIP(PRT), and CND(GWE).  The number of supported packages is expected to increase in the future.  The capability can be turned on with the package EXPORT\_ARRAY\_ASCII option.  The package parameter export set is pre-defined and currently focuses on griddata.  The number of parameters per package may also increase in the future.
	%	\item xxx
	\end{itemize}

	%\underline{EXAMPLES}
	%\begin{itemize}
	%	\item xxx
	%	\item xxx
	%	\item xxx
	%\end{itemize}

	\textbf{\underline{BUG FIXES AND OTHER CHANGES TO EXISTING FUNCTIONALITY}} \\
	\underline{BASIC FUNCTIONALITY}
	\begin{itemize}
		\item When the Adaptive Time Step (ATS) Package was activated, and failed time steps were rerun successfully using a shorter time step, the cumulative budgets reported in the listing files were incorrect.  Cumulative budgets included contributions from the failed time steps.  The program was fixed so that cumulative budgets are not tallied until the time step is finalized.
		\item For multi-model groundwater flow simulations that use the Buoyancy (BUY) Package, variable-density terms were not included, in all cases, along the interface between two GWF models.  The program was corrected so that flows between two GWF models include the variable-density terms when the BUY Package is active.
	%	\item xxx
	\end{itemize}

	%\underline{INTERNAL FLOW PACKAGES}
	%\begin{itemize}
	%	\item xxx
	%	\item xxx
	%	\item xxx
	%\end{itemize}

	\underline{STRESS PACKAGES}
	\begin{itemize}
		\item Floating point overflow errors would occur when negative conductance (COND) or auxiliary multiplier (AUXMULT) values were specified in the Drain, River, and General Head stress packages. This bug was fixed by checking if COND and AUXMULT values are greater than or equal to zero. The program will terminate with and error if negative COND or AUXMULT values are found.
	%	\item xxx
	%	\item xxx
	\end{itemize}

	\underline{ADVANCED STRESS PACKAGES}
	\begin{itemize}
		\item A divide by zero error would occur in the Streamflow Routing package when reaches were deactivated during a simulation. This bug was fixed by checking if the downstream reach is inactive before calculating the flow to the downstream reach.
		\item When using the mover transport (MVT) package with UZF and UZT, rejected infiltration was paired with the calculated concentration of the UZF object rather than the user-specified concentration assigned to the infiltration.  This bug was fixed by instead pairing the rejected infiltration transferred by the MVR and MVT packages with the user-specified concentration assigned in UZTSETTING with the keyword ``INFILTRATION.''  With this change, MODFLOW 6 simulations that use UZF with the ``SIMULATE\_GWSEEP'' option will not transfer this particular source of water with the correct concentration.  Instead, the DRN package should be used to simulate groundwater discharge to land surface.  By simulating groundwater discharge to land surface with the DRN package and rejected infiltration with the UZF package, the correct concentrations will be assigned to the water transferred by the MVR package.
		\item The SIMULATE\_GWSEEP variable in the UZF package OPTIONS block will eventually be deprecated.  The same functionality may be achieved using an option available within the DRN package called AUXDEPTHNAME.  The details of the drainage option are given in the supplemental technical information document (mf6suptechinfo) provided with the release.  Deprecation of the SIMULATE\_GWSEEP option is motivated by the potential for errors noted above.
		\item The Streamflow Routing package would not calculate groundwater discharge to a reach in cases where the groundwater head is above the top of the reach and the inflow to the reach from upstream reaches, specified inflows, rainfall, and runoff is zero. This bug has been fixed by eliminating the requirement that the conductance calculated based on the initial calculated stage is greater than zero in order to solve for groundwater . Instead, .As a result, differences in groundwater and surface-water exchange and groundwater heads in existing models may occur.  
		\item The capability to inactivate lakes (using the STATUS INACTIVE setting) did not work properly for the GWF Lake Package.  The Lake Package was fixed so that inactive lakes have a zero flow value with connected GWF model cells and that the lake stage is assigned the inactive value (1.E30).  The listing, budget, and observation files were modifed to accurately report inactive lakes.
	%	\item xxx
	\end{itemize}

	%\underline{SOLUTION}
	%\begin{itemize}
	%	\item xxx
	%	\item xxx
	%	\item xxx
	%\end{itemize}

	%\underline{EXCHANGES}
	%\begin{itemize}
	%	\item xxx
	%	\item xxx
	%	\item xxx
	%\end{itemize}

